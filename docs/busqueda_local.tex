\section{Heurística de Búsqueda Local}

\subsection{Algoritmo}

Antes de explicar nuestro algoritmo, comenzemos definiendo que es una heurística de búsqueda local. Para cada solución factible $s \in S$, se define $N(s)$ como el conjunto de soluciones vecinas de $s$. Un procedimiento de búsqueda local toma una solución inicial $s$ e iterativamente la mejora reemplazándola por otra solución mejor del conjunto $N(s)$, hasta llegar a un optimo local. El algoritmo se puede ver con el siguiente pseudocodigo:

\begin{algorithmic}
\Procedure{localSearch}{G}
\State{s $\gets$ getInitialSolution(G)}
\State{localSolution $\gets$ true}
\While{localSolution}
	\State{$localSolution \gets false$}
	\ForAll{$\hat{s} \in N(s)$}
		\If{$|\hat{s}| < |s|$}
			\State{$s \gets \hat{s}$}
			\State{$localSolution \gets true$}
			\State{break}
		\EndIf
	\EndFor
\EndWhile
\EndProcedure
\end{algorithmic}

\hspace{1px}

En primer lugar hay que pensar que algoritmo utilizar en la función $getInitialSolution(G)$. Para esto, utilizamos cualquiera de las heurísticas constructivas golosas del paso anterior.

Luego, debemos identificar como construiremos las diferentes $s \in N(s)$, es decir, como construiremos la función que nos devuelve los vecinos de una solución parcial $N(S)$.

\subsection{Vecindades}

Para este algoritmo, utilizaremos los siguientes dos criterios para definir la vecindad de una solución $s$:

\begin{enumerate}
\item \underline{Primera vecindad:}
Para la primera vecindad simplemente tomamos un vértice que actualmente no pertenece a la solución local. Luego, quitamos todos sus vértices adyacentes y verificamos si tenemos una solución con menor cardinal.
\item \underline{Segunda vecindad:}
Para este criterio, lo que hacemos es buscar dos nodos que no pertenecen a la solución local. Los agregamos, quitamos sus nodos adyacentes, y verificamos si el nuevo conjunto es un cubrimiento de menor cardinal.
\end{enumerate}

\subsection{Complejidad}

\subsubsection{Primera vecindad}

En una iteración, el primer algoritmo de vecindad agrega un nodo y luego quita sus adyacentes. Luego verifica que los adyacentes de estos vértices que hemos quitado son alcanzables. Por lo tanto, en el peor caso una iteración tiene orden \order{n \times \Delta(G)^3}. Esto se debe a que se debe verificar que todos los nodos adyacentes a los que saque son adyacentes a algún otro nodo del conjunto en \order{\Delta(G)} para cada nodo adyacente ($\Delta(G)$) a los adyacentes que pude quitar ($\Delta(G)$). Si el nuevo conjunto de nodos no es un CIDM, simplemente restauramos el grafo en \order{\Delta(G)}.

\subsubsection{Segunda vecindad}

En el segundo caso, probamos agregando todos los pares de nodos a la solución actual, quitando sus nodos adyacentes y verificando si luego es una solución. Para ello, simplemente repetimos el procedimiento de la primera vecindad.

Este procedimiento lo repetimos para todo par de $v \not\in S$. Podemos acotar esto de forma grotesca por $\binom{n}{2}$. Por lo tanto la complejidad total de una iteración es de \order{\binom{n}{2} \times \Delta(G)^3}.
