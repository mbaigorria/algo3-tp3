\section{Heurística de Búsqueda Local}

\subsection{Algoritmo}

Antes de explicar nuestro algoritmo, comenzemos definiendo que es una heurística de búsqueda local. Para cada solución factible $s \in S$, se define $N(s)$ como el conjunto de soluciones vecinas de $s$. Un procedimiento de búsqueda local toma una solución inicial $s$ e iterativamente la mejora reemplazándola por otra solución mejor del conjunto $N(s)$, hasta llegar a un óptimo local. El algoritmo se puede ver con el siguiente pseudocódigo:

\begin{algorithmic}
\Procedure{localSearch}{G}
\State{s $\gets$ getInitialSolution(G)}
\State{localSolution $\gets$ true}
\While{localSolution}
	\State{$localSolution \gets false$}
	\ForAll{$\hat{s} \in N(s)$}
		\If{$|\hat{s}| < |s|$}
			\State{$s \gets \hat{s}$}
			\State{$localSolution \gets true$}
			\State{break}
		\EndIf
	\EndFor
\EndWhile
\EndProcedure
\end{algorithmic}

\hspace{1px}

En primer lugar hay que pensar que algoritmo utilizar en la función $getInitialSolution(G)$. Para esto, utilizamos cualquiera de las heurísticas constructivas golosas del paso anterior.

Luego, debemos identificar como construiremos las diferentes $s \in N(s)$, es decir, como construiremos la función que nos devuelve los vecinos de una solución parcial $N(S)$.

\subsection{Vecindades}

Al aplicar este algoritmo a un grafo $G(V,E)$, utilizaremos los siguientes dos criterios para definir la vecindad de una solución $s$:

\begin{enumerate}
\item \underline{Primera vecindad:}
La primera vecindad $N(s)$ esta dada por el conjunto de nodos tal que para algúna solución $n \in N(s)$, $v \in G.V \land v \notin s$, $v \in n$, $adj(v) \not\subset n$, el resto de los nodos en s están en $n$ y $n$ es un conjunto independiente dominante.

\item \underline{Segunda vecindad:}
La segunda vecindad $N(s)$ esta dada por el conjunto de nodos tal que para algúna solución $n \in N(s)$ y un par de vértices $u,v \in G.V \land u,v \notin s$, $u,v \in n$, $adj(u) \cup adj(v) \not\subset n$, el resto de los nodos en s están en n y n es un conjunto independiente dominante.
\end{enumerate}

\subsection{Complejidad}

\subsubsection{Primera vecindad}

\subsubsection*{Pseudocódigo}

\begin{algorithmic}
\Procedure{N}{G,s}
\State{removedNodes $\gets \emptyset$}
\ForAll{v $\in$ G.V}
	\If{v.added == true $\lor$ v.degree == 1}
		\State{continue}
	\EndIf
	
	\State{v.added $\gets$ true}
	
	\ForAll{adjNode $\in$ v.adj}
		\If{adjNode.added == false}
			\State{continue}
		\EndIf
		
		\State{removed.push(adjNode)}
		\State{adjNode.added $\gets$ false}
		
		\ForAll{adjToAdj $\in$ adjNode.adj}
			\If{!isReachable(G, adjToAdj)}
				\While{!removedNodes.isEmpty()}
					\State{n $\gets$ removedNodes.pop()}
					\State{n.added $\gets$ true}
					\State{v.added $\gets$ false}
					\State{tryNextNode()}
				\EndWhile
			\EndIf
		\EndFor
		\State{return}
	\EndFor
	
\EndFor
\EndProcedure

\end{algorithmic}

Lo que hace este procedimiento es buscar un vecino válido. De no encontrarlo, restaura el grafo y prueba el próximo nodo para buscar otro vecino con la función $tryNextNode()$, que es una especie de jump. La función $isReachable(G, node)$ simplemente se fija si dado un nodo existe algún vecino que este en el cubrimiento. Caso contrario, no estamos ante un conjunto válido. Esto lo hace en \order{\Delta(G)}.

\subsubsection*{Complejidad}

En una iteración, el primer algoritmo de vecindad agrega un nodo y luego quita sus adyacentes en \order{\Delta(G)}. Luego verifica que los adyacentes de estos vértices que hemos quitado son alcanzables. Por lo tanto, en el peor caso una iteración tiene orden \order{n \times \Delta(G)^3}. Esto se debe a que se debe verificar que todos los nodos adyacentes a los que saque son adyacentes a algún otro nodo del conjunto en \order{\Delta(G)} para cada nodo adyacente ($\Delta(G)$) a los adyacentes que pude quitar ($\Delta(G)$). Si el nuevo conjunto de nodos no es un CIDM, simplemente restauramos el grafo en \order{\Delta(G)}.

\subsubsection{Segunda vecindad}

\subsubsection*{Pseudocódigo}

\begin{algorithmic}
\Procedure{N}{G,s}
\State{removedNodes $\gets \emptyset$}
\ForAll{(u,v) $\in$ G.V}
	\If{u.added == true $\lor$ u.degree == 1 $\lor$ v.added == true $\lor$ v.degree == 1}
		\State{continue}
	\EndIf
	
	\State{u.added $\gets$ true}
	\State{v.added $\gets$ true}	
	
	\ForAll{adjNode $\in$ v.adj}
		\If{adjNode.added == false}
			\State{continue}
		\EndIf
		
		\State{removed.push(adjNode)}
		\State{adjNode.added $\gets$ false}
		
		\ForAll{adjToAdj $\in$ adjNode.adj}
			\If{!isReachable(G, adjToAdj)}
				\While{!removedNodes.isEmpty()}
					\State{n $\gets$ removedNodes.pop()}
					\State{n.added $\gets$ true}
					\State{u.added $\gets$ false}
					\State{v.added $\gets$ false}
					\State{tryNextPair()}
				\EndWhile
			\EndIf
		\EndFor
	\EndFor
	
	\ForAll{adjNode $\in$ u.adj}
		\If{adjNode.added == false}
			\State{continue}
		\EndIf
		
		\State{removed.push(adjNode)}
		\State{adjNode.added $\gets$ false}
		
		\ForAll{adjToAdj $\in$ adjNode.adj}
			\If{!isReachable(G, adjToAdj)}
				\While{!removedNodes.isEmpty()}
					\State{n $\gets$ removedNodes.pop()}
					\State{u.added $\gets$ false}
					\State{v.added $\gets$ false}
					\State{tryNextPair()}
				\EndWhile
			\EndIf
		\EndFor
		\State{return}
	\EndFor
	
\EndFor
\EndProcedure

\end{algorithmic}

\subsubsection*{Complejidad}

En el segundo caso, probamos agregando todos los pares de nodos a la solución actual, quitando sus nodos adyacentes y verificando si luego es una solución. Para ello, simplemente repetimos el procedimiento de la primera vecindad.

Este procedimiento lo repetimos para todo par de $v \not\in S$. Podemos acotar esto por $\binom{n}{2}$. Por lo tanto la complejidad total de una iteración es de \order{\binom{n}{2} \times \Delta(G)^3}.