\section{Experimentacion}

\subsection{Metodologia}

Para la experimentacion decidimos dividir la misma en dos etapas.\\

\begin{itemize}
	\item \textbf{Comparacion de la heuristica y el algoritmo exacto:} Se generan ciertos casos y se calcula la solucion optima mediante el algoritmo exacto, luego se calculan soluciones mediante las distintas heuristicas para los mismos casos y se analiza la diferencia. Debido a la complejidad del algoritmo exacto, solo se realiza el calculo exacto desde 6 hasta 30 nodos, tomando los casos donde la cantidad de nodos es par.
	\item \textbf{Analisis de velocidad:} Se tomaron casos con una cantidad considerable de nodos para analizar la velocidad de los algoritmos, en base a los resultados obtenidos se decide con 	que variantes quedarnos de cada implementacion para la comparacion final. En este caso se tomaron de 6 hasta 600 nodos, tambien pares.
\end{itemize}

\subsection{Familias de Grafos}

Para experimentar decidimos probar con 3 familias distintas de grafos, estas son:

\begin{itemize}
	\item Grafos aleatorios
	\item Grafos bipartitos completos
	\item Grafos $d$-regulares
\end{itemize}

Una vez probadas las diferentes combinaciones, procedemos a hacer la verificacion final sobre arboles binarios.