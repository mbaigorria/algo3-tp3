\section{Comparacion de eficiencia}

Las familias finalmente escogidas para hacer un nuevo analisis fueron:

\begin{itemize}
	\item Heuristica Contructiva Golosa por Scoring
	\item Busqueda local con segundo criterio de vecinidad, utilizando la Heuristica Golosa por Grado para generar la solucion inicial
	\item GRASP3, con $j = 3$ y $k = 5$
\end{itemize}

Estas fueron las configuracion elegidas producto de la experimentacion realizada anteriormente.

\subsection{Resultados}

Para analizar la eficacia de las configuraciones, nos reservamos una familia de grafos para experimentar, esta es la de union de componentes conexas. La metodologia aplcada fue la misma que en los otros casos, los resultados obtenidos fueron los siguientes:

\begin{figure}[H]
\centering

\begin{subfigure}[b]{0.4\textwidth}
	\includegraphics[scale=0.6]{graph/{output_greedy_1_5_circulos.csvTime}.pdf}
	\begin{center}
	Heuristica Golosa
	\end{center}
\end{subfigure}
\begin{subfigure}[b]{0.4\textwidth}
	\includegraphics[scale=0.6]{graph/{output_greedy_1_5_circulos.csvSolution}.pdf}
	\begin{center}
	Resultado de la Heuristica Golosa
	\end{center}
\end{subfigure}

\begin{subfigure}[b]{0.4\textwidth}
	\includegraphics[scale=0.6]{graph/{output_local_4_5_circulos.csvTime}.pdf}
	\begin{center}
	Busqueda Local
	\end{center}
\end{subfigure}
\begin{subfigure}[b]{0.4\textwidth}
	\includegraphics[scale=0.6]{graph/{output_local_4_5_circulos.csvSolution}.pdf}
	\begin{center}
	Resultado de la Busqueda Local
	\end{center}
\end{subfigure}
\end{figure}

\begin{figure}[H]
\centering

\begin{subfigure}[b]{0.4\textwidth}
	\includegraphics[scale=0.6]{graph/{output_grasp_3_k3_5_circulos.csvTime}.pdf}
	\begin{center}
	GRASP3
	\end{center}
\end{subfigure}
\begin{subfigure}[b]{0.4\textwidth}
	\includegraphics[scale=0.6]{graph/{output_grasp_3_k3_5_circulos.csvSolution}.pdf}
	\begin{center}
	Resultados de GRASP3
	\end{center}
\end{subfigure}
\end{figure}

De las tres configuraciones, podemos ver claramente la Heuristica Golosa por Scoring es la que mejor rendimiento tuvo, no solo temporal sino que ademas el tamaño de las soluciones. Lamentablmente la Busqueda Local y la configuracion de GRASP no tuvieron el mismo rendimiento, sin embargo, creemos que esto no es representativo de la eficiencia de la heuristica GRASP. Tambien consideramos importante destacar que GRASP permite un nivel de personalizion bastante grande, con lo cual no descartamos que exista otra configuracion de GRASP que nos permita mejorar tanto el tiempo de convergencia, como el tamaño de las soluciones.