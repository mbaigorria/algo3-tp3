\section{Comparación de eficiencia}

Las familias finalmente escogidas para hacer un nuevo análisis fueron:

\begin{itemize}
	\item Heuristica Contructiva Golosa por Scoring.
	\item Busqueda local con segundo criterio de vecinidad, utilizando la Heuristica Golosa por Grado para generar la solucion inicial.
	\item GRASP3, con $j = 3$ y $k = 5$.
\end{itemize}

Estas fueron las configuración elegidas producto de la experimentación realizada anteriormente.

\subsection{Resultados}

Para analizar la eficacia de las configuraciones, nos reservamos una familia de grafos para experimentar, esta es la de unión de componentes conexas. Lo hicimos de esta manera para evitar el \texttt{over-fitting}, dado que no podemos evaluar la efectividad de GRASP bajo el mismo \texttt{training set}. La metodología aplicada fue la misma que en los otros casos, los resultados obtenidos fueron los siguientes:

\begin{figure}[H]
\centering

\begin{subfigure}[h]{0.4\textwidth}
	\includegraphics[scale=0.5]{graph/{output_greedy_1_5_circulos.csvTime}.pdf}
	\caption*{Heurística Golosa}
\end{subfigure}
\begin{subfigure}[h]{0.4\textwidth}
	\includegraphics[scale=0.5]{graph/{output_greedy_1_5_circulos.csvSolution}.pdf}
	\caption*{Resultado de la Heurística Golosa}
\end{subfigure}

\begin{subfigure}[h]{0.4\textwidth}
	\includegraphics[scale=0.5]{graph/{output_local_4_5_circulos.csvTime}.pdf}
	\caption*{Búsqueda Local}
\end{subfigure}
\begin{subfigure}[h]{0.4\textwidth}
	\includegraphics[scale=0.5]{graph/{output_local_4_5_circulos.csvSolution}.pdf}
	\caption*{Resultado de la Búsqueda Local}
\end{subfigure}
\end{figure}

\begin{figure}[H]
\centering

\begin{subfigure}[h]{0.4\textwidth}
	\includegraphics[scale=0.5]{graph/{output_grasp_3_k3_5_circulos.csvTime}.pdf}
	\caption*{GRASP3}
\end{subfigure}
\begin{subfigure}[h]{0.4\textwidth}
	\includegraphics[scale=0.5]{graph/{output_grasp_3_k3_5_circulos.csvSolution}.pdf}
	\caption*{Resultados de GRASP3}
\end{subfigure}
\end{figure}

De las tres configuraciones, podemos ver claramente que la Heurística Golosa por Scoring es la que mejor rendimiento tuvo, no solo temporal sino que ademas el tamaño de las soluciones. Lamentablemente la Búsqueda Local y la configuración de GRASP no tuvieron el mismo rendimiento, Sin embargo, creemos que esto no es representativo de la eficiencia de la Heurística GRASP. También consideramos importante destacar que GRASP permite un nivel de personalización bastante grande, con lo cual no descartamos que exista otra configuración de GRASP que nos permita mejorar tanto el tiempo de convergencia, como el tamaño de las soluciones.