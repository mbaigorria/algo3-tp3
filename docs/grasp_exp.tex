\subsection{Experimentación}

Debido a la longitud de los nombres, se decidió numerar las diferentes configuraciones de GRASP:

\begin{table}[H]
\centering
\begin{tabular}{cl|c|c|c|c|c|c|c|c|}
\cline{3-10}
                                                             &                         & \multicolumn{8}{c|}{GRASP}                                                            \\ \cline{3-10} 
                                                             &                         & 1        & 2        & 3        & 4        & 5        & 6        & 7        & 8        \\ \hline
\multicolumn{1}{|c|}{\multirow{2}{*}{Solucion Inicial}}      & Degree Randomized       & $\times$ &          & $\times$ &          & $\times$ &          & $\times$ &          \\ \cline{2-10} 
\multicolumn{1}{|c|}{}                                       & Score Randomized        &          & $\times$ &          & $\times$ &          & $\times$ &          & $\times$ \\ \hline
\multicolumn{1}{|c|}{\multirow{2}{*}{Criterio de Localidad}} & Local Search (1 node)   & $\times$ & $\times$ &          &          & $\times$ & $\times$ &          &          \\ \cline{2-10} 
\multicolumn{1}{|c|}{}                                       & Local Search (2 nodes)  &          &          & $\times$ & $\times$ &          &          & $\times$ & $\times$ \\ \hline
\multicolumn{1}{|c|}{\multirow{2}{*}{Criterio de Parada}}    & Iteraciones            &          &          &          &          & $\times$ & $\times$ & $\times$ & $\times$ \\ \cline{2-10} 
\multicolumn{1}{|c|}{}                                       & Iteraciones sin mejoras & $\times$ & $\times$ & $\times$ & $\times$ &          &          &          &          \\ \hline
\end{tabular}
\caption{Configuraciones utilizadas para GRASP.}
\end{table}

%\begin{itemize}
%	\item GRASP1: Greedy por cantidad, Búsqueda Local con primer criterio de vecinidad, Terminación sin mejoras
%	\item GRASP2: Greedy por valor, Búsqueda Local con primer criterio de vecinidad, Terminación sin mejoras
%	\item GRASP3: Greedy por cantidad, Búsqueda Local con segundo criterio de vecinidad, Terminación sin mejoras
%	\item GRASP4: Greedy por cantidad, Búsqueda Local con segundo criterio de vecinidad, Terminación sin mejoras
%	\item GRASP5: Greedy por valor, Búsqueda Local con primer criterio de vecinidad, Terminación prefijada
%	\item GRASP6: Greedy por cantidad, Búsqueda Local con primer criterio de vecinidad, Terminación prefijada
%	\item GRASP7: Greedy por cantidad, Búsqueda Local con segundo criterio de vecinidad, Terminación prefijada
%	\item GRASP8: Greedy por valor, Búsqueda Local con segundo criterio de vecinidad, Terminación prefijada
%\end{itemize}

%grasp1 = Midsbyiteration, HeapConstructiveRandomized, LocalSearch \\
%grasp2 = Midsbyiteration, HeapConstructiveRandomized2, LocalSearch \\
%grasp3 = Midsbyiteration, HeapConstructiveRandomized, LocalSearch2 \\
%grasp4 = Midsbyiteration, HeapConstructiveRandomized2, LocalSearch2 \\
%grasp5 = Midsbyvalue, HeapConstructiveRandomized, LocalSearch \\
%grasp6 = Midsbyvalue, HeapConstructiveRandomized2, LocalSearch \\
%grasp7 = Midsbyvalue, HeapConstructiveRandomized, LocalSearch2 \\
%grasp8 = Midsbyvalue, HeapConstructiveRandomized2, LocalSearch2 

Para la experimentación se siguió con la metodología indicada anteriormente. Las variaciones en los  métodos  no solo provienen de como elegimos la solución inicial, el criterio de localidad y parada, si no que también de la elección de los parámetros $k$ y $j$, donde $k$ es el parámetro utilizado las funciones de solución inicial randomizadas y $j$ es la cantidad de iteraciones. Para experimentar, en primera instancia decidimos utilizar $(k,j) = (7, 7)$. Los resultados fueron los siguientes.

\subsubsection{GRASP1}

Los resultados temporales obtenidos fueron los siguientes:

\begin{figure}[H]
\centering

\begin{subfigure}[h]{0.4\textwidth}
	\includegraphics[scale=0.5]{graph/{output_grasp_1_1_n.csvTime}.pdf}
	\begin{center}
	Grafos Aleatorios ($m = n$)
	\end{center}
\end{subfigure}
\begin{subfigure}[h]{0.4\textwidth}
	\includegraphics[scale=0.5]{graph/{output_grasp_1_1_2n.csvTime}.pdf}
	\begin{center}
	Grafos Aleatorios ($m = 2n$)
	\end{center}
\end{subfigure}

\begin{subfigure}[h]{0.4\textwidth}
	\includegraphics[scale=0.5]{graph/{output_grasp_1_1_n2.csvTime}.pdf}
	\begin{center}
	Grafos Aleatorios ($m = \frac{n}{2}$)
	\end{center}
\end{subfigure}
\begin{subfigure}[h]{0.4\textwidth}
	\includegraphics[scale=0.5]{graph/{output_grasp_1_3_n4.csvTime}.pdf}
	\begin{center}
	Grafos Bipartitos ($\frac{n}{4}$ nodos en la segunda componente)
	\end{center}
\end{subfigure}

\end{figure}

\begin{figure}[H]
\centering

\begin{subfigure}[h]{0.4\textwidth}
	\includegraphics[scale=0.5]{graph/{output_grasp_1_3_3n4.csvTime}.pdf}
	\begin{center}
	Grafos Bipartitos ($\frac{3n}{4}$ nodos en la segunda componente)
	\end{center}
\end{subfigure}
\begin{subfigure}[h]{0.4\textwidth}
	\includegraphics[scale=0.5]{graph/{output_grasp_1_2_n4.csvTime}.pdf}
	\begin{center}
	Grafos $d$-regulares ($d = \frac{n}{4}$)
	\end{center}
\end{subfigure}

\begin{subfigure}[h]{0.4\textwidth}
	\includegraphics[scale=0.5]{graph/{output_grasp_1_2_n2.csvTime}.pdf}
	\begin{center}
	Grafos $d$-regulares ($m = \frac{n}{2}$)
	\end{center}
\end{subfigure}
\begin{subfigure}[h]{0.4\textwidth}
	\includegraphics[scale=0.5]{graph/{output_grasp_1_2_3n4.csvTime}.pdf}
	\begin{center}
	Grafos $d$-regulares ($m = \frac{3n}{4}$)
	\end{center}
\end{subfigure}

\begin{subfigure}[h]{0.4\textwidth}
	\includegraphics[scale=0.5]{graph/{output_grasp_1_4_arbol.csvTime}.pdf}
	\begin{center}
	Árboles Binarios
	\end{center}
\end{subfigure}
\begin{subfigure}[h]{0.4\textwidth}
	\includegraphics[scale=0.5]{graph/{output_grasp_1_1_clique.csvTime}.pdf}
	\begin{center}
	Clique
	\end{center}
\end{subfigure}
\end{figure}

Para el análisis del tamaño de la solución, vamos a ver los resultados por cada familia. En el caso de los aleatorios, los resultados para estas configuraciones fueron los siguiente:

\begin{table}[H]
\centering
\label{my-label}
\begin{tabular}{|l|lll|}
\hline
        & \multicolumn{1}{l|}{m = n/2} & \multicolumn{1}{l|}{m = n} & m = 2n \\ \hline
n = 40  & 26                           & 21                         & 15     \\ \cline{1-1}
n = 60  & 39                           & 32                         & 21     \\ \cline{1-1}
n = 80  & 49                           & 34                         & 25     \\ \cline{1-1}
n = 100 & 60                           & 49                         & 32     \\ \cline{1-1}
n = 120 & 75                           & 55                         & 39     \\ \hline
\end{tabular}
\caption{Grafos aleatorios. Los números en la tabla muestran la cantidad de vértices en el conjunto dominante independiente final.}
\end{table}

Para los Árboles $d$-regulares y Árboles Binarios, la heurística GRASP comenzó a mostrar resultados poco eficientes, eligiendo una mayor cantidad de nodos que la habitual. Esta tendencia se da para todas las configuraciones de GRASP.

\subsubsection{GRASP2}

Los resultados temporales obtenidos fueron los siguientes:

\begin{figure}[H]
\centering

\begin{subfigure}[h]{0.4\textwidth}
	\includegraphics[scale=0.5]{graph/{output_grasp_2_1_n.csvTime}.pdf}
	\begin{center}
	Grafos Aleatorios ($m = n$)
	\end{center}
\end{subfigure}
\begin{subfigure}[h]{0.4\textwidth}
	\includegraphics[scale=0.5]{graph/{output_grasp_2_1_2n.csvTime}.pdf}
	\begin{center}
	Grafos Aleatorios ($m = 2n$)
	\end{center}
\end{subfigure}

\begin{subfigure}[h]{0.4\textwidth}
	\includegraphics[scale=0.5]{graph/{output_grasp_2_1_n2.csvTime}.pdf}
	\begin{center}
	Grafos Aleatorios ($m = \frac{n}{2}$)
	\end{center}
\end{subfigure}
\begin{subfigure}[h]{0.4\textwidth}
	\includegraphics[scale=0.5]{graph/{output_grasp_2_3_n4.csvTime}.pdf}
	\begin{center}
	Grafos Bipartitos ($\frac{n}{4}$ nodos en la segunda componente)
	\end{center}
\end{subfigure}
\end{figure}

\begin{figure}[H]
\centering

\begin{subfigure}[h]{0.4\textwidth}
	\includegraphics[scale=0.5]{graph/{output_grasp_2_3_3n4.csvTime}.pdf}
	\begin{center}
	Grafos Bipartitos ($\frac{3n}{4}$ nodos en la segunda componente)
	\end{center}
\end{subfigure}
\begin{subfigure}[h]{0.4\textwidth}
	\includegraphics[scale=0.5]{graph/{output_grasp_2_2_n4.csvTime}.pdf}
	\begin{center}
	Grafos $d$-regulares ($d = \frac{n}{4}$)
	\end{center}
\end{subfigure}

\begin{subfigure}[h]{0.4\textwidth}
	\includegraphics[scale=0.5]{graph/{output_grasp_2_2_n2.csvTime}.pdf}
	\begin{center}
	Grafos $d$-regulares ($m = \frac{n}{2}$)
	\end{center}
\end{subfigure}
\begin{subfigure}[h]{0.4\textwidth}
	\includegraphics[scale=0.5]{graph/{output_grasp_2_2_3n4.csvTime}.pdf}
	\begin{center}
	Grafos $d$-regulares ($m = \frac{3n}{4}$)
	\end{center}
\end{subfigure}

\begin{subfigure}[h]{0.4\textwidth}
	\includegraphics[scale=0.5]{graph/{output_grasp_2_4_arbol.csvTime}.pdf}
	\begin{center}
	Árboles Binarios
	\end{center}
\end{subfigure}
\begin{subfigure}[h]{0.4\textwidth}
	\includegraphics[scale=0.5]{graph/{output_grasp_2_1_clique.csvTime}.pdf}
	\begin{center}
	Clique
	\end{center}
\end{subfigure}
\end{figure}

Para el análisis del tamaño de la solución, vamos a ver los resultados por cada familia. En el caso de los aleatorios, los resultados para estas configuraciones fueron los siguiente:

\begin{table}[H]
\centering
\label{my-label}
\begin{tabular}{|l|lll|}
\hline
        & \multicolumn{1}{l|}{m = n/2} & \multicolumn{1}{l|}{m = n} & m = 2n \\ \hline
n = 40  & 29                           & 23                         & 15     \\ \cline{1-1}
n = 60  & 40                           & 34                         & 22     \\ \cline{1-1}
n = 80  & 53                           & 39                         & 29     \\ \cline{1-1}
n = 100 & 62                           & 51                         & 34     \\ \cline{1-1}
n = 120 & 80                           & 59                         & 41     \\ \hline
\end{tabular}
\caption{Grafos aleatorios. Los números en la tabla muestran la cantidad de vértices en el conjunto dominante independiente final.}
\end{table}

\subsubsection{GRASP3}

Los resultados temporales obtenidos fueron los siguientes:

\begin{figure}[H]
\centering

\begin{subfigure}[h]{0.4\textwidth}
	\includegraphics[scale=0.5]{graph/{output_grasp_3_1_n.csvTime}.pdf}
	\begin{center}
	Grafos Aleatorios ($m = n$)
	\end{center}
\end{subfigure}
\begin{subfigure}[h]{0.4\textwidth}
	\includegraphics[scale=0.5]{graph/{output_grasp_3_1_2n.csvTime}.pdf}
	\begin{center}
	Grafos Aleatorios ($m = 2n$)
	\end{center}
\end{subfigure}

\begin{subfigure}[h]{0.4\textwidth}
	\includegraphics[scale=0.5]{graph/{output_grasp_3_1_n2.csvTime}.pdf}
	\begin{center}
	Grafos Aleatorios ($m = \frac{n}{2}$)
	\end{center}
\end{subfigure}
\begin{subfigure}[h]{0.4\textwidth}
	\includegraphics[scale=0.5]{graph/{output_grasp_3_3_n4.csvTime}.pdf}
	\begin{center}
	Grafos Bipartitos ($\frac{n}{4}$ nodos en la segunda componente)
	\end{center}
\end{subfigure}
\end{figure}

\begin{figure}[H]
\centering

\begin{subfigure}[h]{0.4\textwidth}
	\includegraphics[scale=0.5]{graph/{output_grasp_3_3_3n4.csvTime}.pdf}
	\begin{center}
	Grafos Bipartitos ($\frac{3n}{4}$ nodos en la segunda componente)
	\end{center}
\end{subfigure}
\begin{subfigure}[h]{0.4\textwidth}
	\includegraphics[scale=0.5]{graph/{output_grasp_3_2_n4.csvTime}.pdf}
	\begin{center}
	Grafos $d$-regulares ($d = \frac{n}{4}$)
	\end{center}
\end{subfigure}

\begin{subfigure}[h]{0.4\textwidth}
	\includegraphics[scale=0.5]{graph/{output_grasp_3_2_n2.csvTime}.pdf}
	\begin{center}
	Grafos $d$-regulares ($m = \frac{n}{2}$)
	\end{center}
\end{subfigure}
\begin{subfigure}[h]{0.4\textwidth}
	\includegraphics[scale=0.5]{graph/{output_grasp_3_2_3n4.csvTime}.pdf}
	\begin{center}
	Grafos $d$-regulares ($m = \frac{3n}{4}$)
	\end{center}
\end{subfigure}

\begin{subfigure}[h]{0.4\textwidth}
	\includegraphics[scale=0.5]{graph/{output_grasp_3_4_arbol.csvTime}.pdf}
	\begin{center}
	Árboles Binarios
	\end{center}
\end{subfigure}
\begin{subfigure}[h]{0.4\textwidth}
	\includegraphics[scale=0.5]{graph/{output_grasp_3_1_clique.csvTime}.pdf}
	\begin{center}
	Clique
	\end{center}
\end{subfigure}
\end{figure}

Para el análisis del tamaño de la solución, vamos a ver los resultados por cada familia. En el caso de los aleatorios, los resultados para estas configuraciones fueron los siguiente:

\begin{table}[H]
\centering
\label{my-label}
\begin{tabular}{|l|lll|}
\hline
        & \multicolumn{1}{l|}{m = n/2} & \multicolumn{1}{l|}{m = n} & m = 2n \\ \hline
n = 40  & 26                           & 21                         & 14     \\ \cline{1-1}
n = 60  & 39                           & 32                         & 19     \\ \cline{1-1}
n = 80  & 49                           & 34                         & 24     \\ \cline{1-1}
n = 100 & 60                           & 47                         & 30     \\ \cline{1-1}
n = 120 & 75                           & 55                         & 37     \\ \hline
\end{tabular}
\caption{Grafos aleatorios. Los números en la tabla muestran la cantidad de vértices en el conjunto dominante independiente final.}
\end{table}

\subsubsection{GRASP4}

Los resultados temporales obtenidos fueron los siguientes:

\begin{figure}[H]
\centering

\begin{subfigure}[h]{0.4\textwidth}
	\includegraphics[scale=0.5]{graph/{output_grasp_4_1_n.csvTime}.pdf}
	\begin{center}
	Grafos Aleatorios ($m = n$)
	\end{center}
\end{subfigure}
\begin{subfigure}[h]{0.4\textwidth}
	\includegraphics[scale=0.5]{graph/{output_grasp_4_1_2n.csvTime}.pdf}
	\begin{center}
	Grafos Aleatorios ($m = 2n$)
	\end{center}
\end{subfigure}

\begin{subfigure}[h]{0.4\textwidth}
	\includegraphics[scale=0.5]{graph/{output_grasp_4_1_n2.csvTime}.pdf}
	\begin{center}
	Grafos Aleatorios ($m = \frac{n}{2}$)
	\end{center}
\end{subfigure}
\begin{subfigure}[h]{0.4\textwidth}
	\includegraphics[scale=0.5]{graph/{output_grasp_4_3_n4.csvTime}.pdf}
	\begin{center}
	Grafos Bipartitos ($\frac{n}{4}$ nodos en la segunda componente)
	\end{center}
\end{subfigure}
\end{figure}

\begin{figure}[H]
\centering

\begin{subfigure}[h]{0.4\textwidth}
	\includegraphics[scale=0.5]{graph/{output_grasp_4_3_3n4.csvTime}.pdf}
	\begin{center}
	Grafos Bipartitos ($\frac{3n}{4}$ nodos en la segunda componente)
	\end{center}
\end{subfigure}
\begin{subfigure}[h]{0.4\textwidth}
	\includegraphics[scale=0.5]{graph/{output_grasp_4_2_n4.csvTime}.pdf}
	\begin{center}
	Grafos $d$-regulares ($d = \frac{n}{4}$)
	\end{center}
\end{subfigure}

\begin{subfigure}[h]{0.4\textwidth}
	\includegraphics[scale=0.5]{graph/{output_grasp_4_2_n2.csvTime}.pdf}
	\begin{center}
	Grafos $d$-regulares ($m = \frac{n}{2}$)
	\end{center}
\end{subfigure}
\begin{subfigure}[h]{0.4\textwidth}
	\includegraphics[scale=0.5]{graph/{output_grasp_4_2_3n4.csvTime}.pdf}
	\begin{center}
	Grafos $d$-regulares ($m = \frac{3n}{4}$)
	\end{center}
\end{subfigure}

\begin{subfigure}[h]{0.4\textwidth}
	\includegraphics[scale=0.5]{graph/{output_grasp_4_4_arbol.csvTime}.pdf}
	\begin{center}
	Árboles Binarios
	\end{center}
\end{subfigure}
\begin{subfigure}[h]{0.4\textwidth}
	\includegraphics[scale=0.5]{graph/{output_grasp_4_1_clique.csvTime}.pdf}
	\begin{center}
	Clique
	\end{center}
\end{subfigure}
\end{figure}

Para el análisis del tamaño de la solución, vamos a ver los resultados por cada familia. En el caso de los aleatorios, los resultados para estas configuraciones fueron los siguiente:

\begin{table}[H]
\centering
\label{my-label}
\begin{tabular}{|l|lll|}
\hline
        & \multicolumn{1}{l|}{m = n/2} & \multicolumn{1}{l|}{m = n} & m = 2n \\ \hline
n = 40  & 25                           & 19                         & 13     \\ \cline{1-1}
n = 60  & 40                           & 32                         & 18     \\ \cline{1-1}
n = 80  & 53                           & 39                         & 26     \\ \cline{1-1}
n = 100 & 60                           & 49                         & 31     \\ \cline{1-1}
n = 120 & 76                           & 55                         & 39     \\ \hline
\end{tabular}
\caption{Grafos aleatorios. Los números en la tabla muestran la cantidad de vértices en el conjunto dominante independiente final.}
\end{table}

\subsubsection{GRASP5}

Los resultados temporales obtenidos fueron los siguientes:

\begin{figure}[H]
\centering

\begin{subfigure}[h]{0.4\textwidth}
	\includegraphics[scale=0.5]{graph/{output_grasp_5_1_n.csvTime}.pdf}
	\begin{center}
	Grafos Aleatorios ($m = n$)
	\end{center}
\end{subfigure}
\begin{subfigure}[h]{0.4\textwidth}
	\includegraphics[scale=0.5]{graph/{output_grasp_5_1_2n.csvTime}.pdf}
	\begin{center}
	Grafos Aleatorios ($m = 2n$)
	\end{center}
\end{subfigure}

\begin{subfigure}[h]{0.4\textwidth}
	\includegraphics[scale=0.5]{graph/{output_grasp_5_1_n2.csvTime}.pdf}
	\begin{center}
	Grafos Aleatorios ($m = \frac{n}{2}$)
	\end{center}
\end{subfigure}
\begin{subfigure}[h]{0.4\textwidth}
	\includegraphics[scale=0.5]{graph/{output_grasp_5_3_n4.csvTime}.pdf}
	\begin{center}
	Grafos Bipartitos ($\frac{n}{4}$ nodos en la segunda componente)
	\end{center}
\end{subfigure}
\end{figure}

\begin{figure}[H]
\centering

\begin{subfigure}[h]{0.4\textwidth}
	\includegraphics[scale=0.5]{graph/{output_grasp_5_3_3n4.csvTime}.pdf}
	\begin{center}
	Grafos Bipartitos ($\frac{3n}{4}$ nodos en la segunda componente)
	\end{center}
\end{subfigure}
\begin{subfigure}[h]{0.4\textwidth}
	\includegraphics[scale=0.5]{graph/{output_grasp_5_2_n4.csvTime}.pdf}
	\begin{center}
	Grafos $d$-regulares ($d = \frac{n}{4}$)
	\end{center}
\end{subfigure}

\begin{subfigure}[h]{0.4\textwidth}
	\includegraphics[scale=0.5]{graph/{output_grasp_5_2_n2.csvTime}.pdf}
	\begin{center}
	Grafos $d$-regulares ($m = \frac{n}{2}$)
	\end{center}
\end{subfigure}
\begin{subfigure}[h]{0.4\textwidth}
	\includegraphics[scale=0.5]{graph/{output_grasp_5_2_3n4.csvTime}.pdf}
	\begin{center}
	Grafos $d$-regulares ($m = \frac{3n}{4}$)
	\end{center}
\end{subfigure}

\begin{subfigure}[h]{0.4\textwidth}
	\includegraphics[scale=0.5]{graph/{output_grasp_5_4_arbol.csvTime}.pdf}
	\begin{center}
	Árboles Binarios
	\end{center}
\end{subfigure}
\begin{subfigure}[h]{0.4\textwidth}
	\includegraphics[scale=0.5]{graph/{output_grasp_5_1_clique.csvTime}.pdf}
	\begin{center}
	Clique
	\end{center}
\end{subfigure}
\end{figure}

Para el análisis del tamaño de la solución, vamos a ver los resultados por cada familia. En el caso de los aleatorios, los resultados para estas configuraciones fueron los siguiente:

\begin{table}[H]
\centering
\label{my-label}
\begin{tabular}{|l|lll|}
\hline
        & \multicolumn{1}{l|}{m = n/2} & \multicolumn{1}{l|}{m = n} & m = 2n \\ \hline
n = 40  & 26                           & 22                         & 15     \\ \cline{1-1}
n = 60  & 40                           & 32                         & 20     \\ \cline{1-1}
n = 80  & 49                           & 35                         & 27     \\ \cline{1-1}
n = 100 & 60                           & 50                         & 35     \\ \cline{1-1}
n = 120 & 75                           & 56                         & 39     \\ \hline
\end{tabular}
\caption{Grafos aleatorios. Los números en la tabla muestran la cantidad de vértices en el conjunto dominante independiente final.}
\end{table}

\subsubsection{GRASP6}

Los resultados temporales obtenidos fueron los siguientes:

\begin{figure}[H]
\centering

\begin{subfigure}[h]{0.4\textwidth}
	\includegraphics[scale=0.5]{graph/{output_grasp_6_1_n.csvTime}.pdf}
	\begin{center}
	Grafos Aleatorios ($m = n$)
	\end{center}
\end{subfigure}
\begin{subfigure}[h]{0.4\textwidth}
	\includegraphics[scale=0.5]{graph/{output_grasp_6_1_2n.csvTime}.pdf}
	\begin{center}
	Grafos Aleatorios ($m = 2n$)
	\end{center}
\end{subfigure}

\begin{subfigure}[h]{0.4\textwidth}
	\includegraphics[scale=0.5]{graph/{output_grasp_6_1_n2.csvTime}.pdf}
	\begin{center}
	Grafos Aleatorios ($m = \frac{n}{2}$)
	\end{center}
\end{subfigure}
\begin{subfigure}[h]{0.4\textwidth}
	\includegraphics[scale=0.5]{graph/{output_grasp_6_3_n4.csvTime}.pdf}
	\begin{center}
	Grafos Bipartitos ($\frac{n}{4}$ nodos en la segunda componente)
	\end{center}
\end{subfigure}
\end{figure}

\begin{figure}[H]
\centering

\begin{subfigure}[h]{0.4\textwidth}
	\includegraphics[scale=0.5]{graph/{output_grasp_6_3_3n4.csvTime}.pdf}
	\begin{center}
	Grafos Bipartitos ($\frac{3n}{4}$ nodos en la segunda componente)
	\end{center}
\end{subfigure}
\begin{subfigure}[h]{0.4\textwidth}
	\includegraphics[scale=0.5]{graph/{output_grasp_6_2_n4.csvTime}.pdf}
	\begin{center}
	Grafos $d$-regulares ($d = \frac{n}{4}$)
	\end{center}
\end{subfigure}

\begin{subfigure}[h]{0.4\textwidth}
	\includegraphics[scale=0.5]{graph/{output_grasp_6_2_n2.csvTime}.pdf}
	\begin{center}
	Grafos $d$-regulares ($m = \frac{n}{2}$)
	\end{center}
\end{subfigure}
\begin{subfigure}[h]{0.4\textwidth}
	\includegraphics[scale=0.5]{graph/{output_grasp_6_2_3n4.csvTime}.pdf}
	\begin{center}
	Grafos $d$-regulares ($m = \frac{3n}{4}$)
	\end{center}
\end{subfigure}

\begin{subfigure}[h]{0.4\textwidth}
	\includegraphics[scale=0.5]{graph/{output_grasp_6_4_arbol.csvTime}.pdf}
	\begin{center}
	Árboles Binarios
	\end{center}
\end{subfigure}
\begin{subfigure}[h]{0.4\textwidth}
	\includegraphics[scale=0.5]{graph/{output_grasp_6_1_clique.csvTime}.pdf}
	\begin{center}
	Clique
	\end{center}
\end{subfigure}
\end{figure}

Para el análisis del tamaño de la solución, vamos a ver los resultados por cada familia. En el caso de los aleatorios, los resultados para estas configuraciones fueron los siguiente:

\begin{table}[H]
\centering
\label{my-label}
\begin{tabular}{|l|lll|}
\hline
        & \multicolumn{1}{l|}{m = n/2} & \multicolumn{1}{l|}{m = n} & m = 2n \\ \hline
n = 40  & 29                           & 21                         & 15     \\ \cline{1-1}
n = 60  & 40                           & 31                         & 21     \\ \cline{1-1}
n = 80  & 53                           & 36                         & 27     \\ \cline{1-1}
n = 100 & 62                           & 50                         & 37     \\ \cline{1-1}
n = 120 & 81                           & 56                         & 41     \\ \hline
\end{tabular}
\caption{Grafos aleatorios. Los números en la tabla muestran la cantidad de vértices en el conjunto dominante independiente final.}
\end{table}

\subsubsection{GRASP7}

Los resultados temporales obtenidos fueron los siguientes:

\begin{figure}[H]
\centering

\begin{subfigure}[h]{0.4\textwidth}
	\includegraphics[scale=0.5]{graph/{output_grasp_7_1_n.csvTime}.pdf}
	\begin{center}
	Grafos Aleatorios ($m = n$)
	\end{center}
\end{subfigure}
\begin{subfigure}[h]{0.4\textwidth}
	\includegraphics[scale=0.5]{graph/{output_grasp_7_1_2n.csvTime}.pdf}
	\begin{center}
	Grafos Aleatorios ($m = 2n$)
	\end{center}
\end{subfigure}

\begin{subfigure}[h]{0.4\textwidth}
	\includegraphics[scale=0.5]{graph/{output_grasp_7_1_n2.csvTime}.pdf}
	\begin{center}
	Grafos Aleatorios ($m = \frac{n}{2}$)
	\end{center}
\end{subfigure}
\begin{subfigure}[h]{0.4\textwidth}
	\includegraphics[scale=0.5]{graph/{output_grasp_7_3_n4.csvTime}.pdf}
	\begin{center}
	Grafos Bipartitos ($\frac{n}{4}$ nodos en la segunda componente)
	\end{center}
\end{subfigure}
\end{figure}

\begin{figure}[H]
\centering

\begin{subfigure}[h]{0.4\textwidth}
	\includegraphics[scale=0.5]{graph/{output_grasp_7_3_3n4.csvTime}.pdf}
	\begin{center}
	Grafos Bipartitos ($\frac{3n}{4}$ nodos en la segunda componente)
	\end{center}
\end{subfigure}
\begin{subfigure}[h]{0.4\textwidth}
	\includegraphics[scale=0.5]{graph/{output_grasp_7_2_n4.csvTime}.pdf}
	\begin{center}
	Grafos $d$-regulares ($d = \frac{n}{4}$)
	\end{center}
\end{subfigure}

\begin{subfigure}[h]{0.4\textwidth}
	\includegraphics[scale=0.5]{graph/{output_grasp_7_2_n2.csvTime}.pdf}
	\begin{center}
	Grafos $d$-regulares ($m = \frac{n}{2}$)
	\end{center}
\end{subfigure}
\begin{subfigure}[h]{0.4\textwidth}
	\includegraphics[scale=0.5]{graph/{output_grasp_7_2_3n4.csvTime}.pdf}
	\begin{center}
	Grafos $d$-regulares ($m = \frac{3n}{4}$)
	\end{center}
\end{subfigure}

\begin{subfigure}[h]{0.4\textwidth}
	\includegraphics[scale=0.5]{graph/{output_grasp_7_4_arbol.csvTime}.pdf}
	\begin{center}
	Árboles Binarios
	\end{center}
\end{subfigure}
\begin{subfigure}[h]{0.4\textwidth}
	\includegraphics[scale=0.5]{graph/{output_grasp_7_1_clique.csvTime}.pdf}
	\begin{center}
	Clique
	\end{center}
\end{subfigure}
\end{figure}

Para el análisis del tamaño de la solución, vamos a ver los resultados por cada familia. En el caso de los aleatorios, los resultados para estas configuraciones fueron los siguiente:

\begin{table}[H]
\centering
\label{my-label}
\begin{tabular}{|l|lll|}
\hline
        & \multicolumn{1}{l|}{m = n/2} & \multicolumn{1}{l|}{m = n} & m = 2n \\ \hline
n = 40  & 26                           & 22                         & 14     \\ \cline{1-1}
n = 60  & 40                           & 31                         & 18     \\ \cline{1-1}
n = 80  & 49                           & 35                         & 26     \\ \cline{1-1}
n = 100 & 60                           & 50                         & 34     \\ \cline{1-1}
n = 120 & 75                           & 56                         & 38     \\ \hline
\end{tabular}
\caption{Grafos aleatorios. Los números en la tabla muestran la cantidad de vértices en el conjunto dominante independiente final.}
\end{table}

\subsubsection{GRASP8}

Los resultados temporales obtenidos fueron los siguientes:

\begin{figure}[H]
\centering

\begin{subfigure}[h]{0.4\textwidth}
	\includegraphics[scale=0.5]{graph/{output_grasp_8_1_n.csvTime}.pdf}
	\begin{center}
	Grafos Aleatorios ($m = n$)
	\end{center}
\end{subfigure}
\begin{subfigure}[h]{0.4\textwidth}
	\includegraphics[scale=0.5]{graph/{output_grasp_8_1_2n.csvTime}.pdf}
	\begin{center}
	Grafos Aleatorios ($m = 2n$)
	\end{center}
\end{subfigure}

\begin{subfigure}[h]{0.4\textwidth}
	\includegraphics[scale=0.5]{graph/{output_grasp_8_1_n2.csvTime}.pdf}
	\begin{center}
	Grafos Aleatorios ($m = \frac{n}{2}$)
	\end{center}
\end{subfigure}
\begin{subfigure}[h]{0.4\textwidth}
	\includegraphics[scale=0.5]{graph/{output_grasp_8_3_n4.csvTime}.pdf}
	\begin{center}
	Grafos Bipartitos ($\frac{n}{4}$ nodos en la segunda componente)
	\end{center}
\end{subfigure}
\end{figure}

\begin{figure}[H]
\centering

\begin{subfigure}[h]{0.4\textwidth}
	\includegraphics[scale=0.5]{graph/{output_grasp_8_3_3n4.csvTime}.pdf}
	\begin{center}
	Grafos Bipartitos ($\frac{3n}{4}$ nodos en la segunda componente)
	\end{center}
\end{subfigure}
\begin{subfigure}[h]{0.4\textwidth}
	\includegraphics[scale=0.5]{graph/{output_grasp_8_2_n4.csvTime}.pdf}
	\begin{center}
	Grafos $d$-regulares ($d = \frac{n}{4}$)
	\end{center}
\end{subfigure}

\begin{subfigure}[h]{0.4\textwidth}
	\includegraphics[scale=0.5]{graph/{output_grasp_8_2_n2.csvTime}.pdf}
	\begin{center}
	Grafos $d$-regulares ($m = \frac{n}{2}$)
	\end{center}
\end{subfigure}
\begin{subfigure}[h]{0.4\textwidth}
	\includegraphics[scale=0.5]{graph/{output_grasp_8_2_3n4.csvTime}.pdf}
	\begin{center}
	Grafos $d$-regulares ($m = \frac{3n}{4}$)
	\end{center}
\end{subfigure}

\begin{subfigure}[h]{0.4\textwidth}
	\includegraphics[scale=0.5]{graph/{output_grasp_8_4_arbol.csvTime}.pdf}
	\begin{center}
	Árboles Binarios
	\end{center}
\end{subfigure}
\begin{subfigure}[h]{0.4\textwidth}
	\includegraphics[scale=0.5]{graph/{output_grasp_8_1_clique.csvTime}.pdf}
	\begin{center}
	Clique
	\end{center}
\end{subfigure}
\end{figure}

Para el análisis del tamaño de la solución, vamos a ver los resultados por cada familia. En el caso de los aleatorios, los resultados para estas configuraciones fueron los siguiente:

\begin{table}[H]
\centering
\label{my-label}
\begin{tabular}{|l|lll|}
\hline
        & \multicolumn{1}{l|}{m = n/2} & \multicolumn{1}{l|}{m = n} & m = 2n \\ \hline
n = 40  & 25                           & 21                         & 16     \\ \cline{1-1}
n = 60  & 40                           & 31                         & 20     \\ \cline{1-1}
n = 80  & 53                           & 36                         & 26     \\ \cline{1-1}
n = 100 & 60                           & 48                         & 34     \\ \cline{1-1}
n = 120 & 76                           & 54                         & 38     \\ \hline
\end{tabular}
\caption{Grafos aleatorios. Los números en la tabla muestran la cantidad de vértices en el conjunto dominante independiente final.}
\end{table}

\subsubsection{Conclusión}

Primero vamos a ver los resultados por cada familia.

\begin{itemize}
	\item Grafos Aleatorios: Para esta familia los resultados fueron variados, y muchos de ellos pudieron mejorar por un margen amplio a las otras heurísticas vistas con anterioridad, sin tomar un tiempo adicional demasiado grande. Los mejores resultados observados en termino de calidad de soluciones es el de GRASP3, que no solo dio mejor solución en casi todos los casos, sino que ademas fue de las más veloces.	
	\item Grafos Bipartitos: Los tiempos de ejecución para todas las instancias de GRASP fueron en general bastante elevados para estos casos. Lamentablemente la calidad de las soluciones variaron bastante respecto a las otras heurísticas, la tendencia entre las diferentes implementaciones de todas formas fue muy marcada.
	\item Grafos $d$-regulares: Esta familia no tuvo buen rendimiento con las diferentes versiones de GRASP. También los resultados obtenidos fueron peores que con las otras heurísticas implementadas anteriormente.
	\item Árboles binarios: Al igual que con las otras heurísticas, el tiempo que tomo resolver cada uno de los grafos no fue constante. Respecto al tamaño de las soluciones, los resultados obtenidos tendían a alejarse de los valores ideales.
	\item Cliques: La resolución de de las cliques tomo una cantidad de tiempo importante a medida que aumentaba la cantidad de nodos en el grafo, esto se dio en todas las configuraciones, principalmente en GRASP3, GRASP4, GRASP7 y GRASP8.
\end{itemize}

Las heurísticas GRASP demostraron que había un gran margen de mejora para los grafos aleatorios. Los resultados obtenidos fueron en su mayoría mejores que los conseguidos aplicando las heurísticas anteriores. Un punto importante a destacar es que el tiempo que tomo la resolución no fue mucho mayor al de las otras heurísticas. Lamentablemente, la eficiencia de las diferentes configuraciones de GRASP no fueron tan significativas para otras familias. Esto se debe a que las heurísticas golosas llegaban al resultado optimo rápidamente, y al seguir intentando buscar mejores soluciones terminamos agregando un overhead.

A pesar de todo esto, consideramos que la mejor configuración fue GRASP3, ya que si bien esta no tuvo un buen rendimiento con las familias que no sean la aleatoria, ninguna configuración fue particularmente buena para el resto de las familias. Valoramos el caso aleatorio principalmente, ya que consideramos que es el que mas chances tenemos de encontrar en un caso real.

\subsection{Calibración de parámetros}

Una vez elegida la configuración, se procedió con la calibración de parámetros. Aquí hay claramente un \texttt{trade-off} entre tiempo de ejecución y calidad de la solución. Primero vamos a ver los resultados obtenidos de variar el valor de $j$, manteniendo el de $k$ en 5. Para probar cada uno de los parámetros se probo con Grafos Aleatorios (con $m = 2n$), para analizar si era posible mejorar el tiempo de convergencia. Los resultados fueron:

\begin{figure}[H]
\centering

\begin{subfigure}[b]{0.4\textwidth}
	\includegraphics[scale=0.5]{graph/{output_grasp_3_k1_1_2n.csvTime}.pdf}
	\begin{center}
	$j = 1$
	\end{center}
\end{subfigure}
\begin{subfigure}[b]{0.4\textwidth}
	\includegraphics[scale=0.5]{graph/{output_grasp_3_k2_1_2n.csvTime}.pdf}
	\begin{center}
	$j = 2$
	\end{center}
\end{subfigure}

\begin{subfigure}[b]{0.4\textwidth}
	\includegraphics[scale=0.5]{graph/{output_grasp_3_k3_1_2n.csvTime}.pdf}
	\begin{center}
	$j = 3$
	\end{center}
\end{subfigure}
\begin{subfigure}[b]{0.4\textwidth}
	\includegraphics[scale=0.5]{graph/{output_grasp_3_k4_1_2n.csvTime}.pdf}
	\begin{center}
	$j = 4$
	\end{center}
\end{subfigure}

\begin{subfigure}[b]{0.4\textwidth}
	\includegraphics[scale=0.5]{graph/{output_grasp_3_k5_1_2n.csvTime}.pdf}
	\begin{center}
	$j = 5$
	\end{center}
\end{subfigure}
\begin{subfigure}[b]{0.4\textwidth}
	\includegraphics[scale=0.5]{graph/{output_grasp_3_k6_1_2n.csvTime}.pdf}
	\begin{center}
	$j = 6$
	\end{center}
\end{subfigure}

\end{figure}

\begin{figure}[H]
\centering

\begin{subfigure}[b]{0.4\textwidth}
	\includegraphics[scale=0.5]{graph/{output_grasp_3_k7_1_2n.csvTime}.pdf}
	\begin{center}
	$j = 7$
	\end{center}
\end{subfigure}
\begin{subfigure}[b]{0.4\textwidth}
	\includegraphics[scale=0.5]{graph/{output_grasp_3_k8_1_2n.csvTime}.pdf}
	\begin{center}
	$j = 8$
	\end{center}
\end{subfigure}
\end{figure}

\begin{figure}[H]
\centering

\begin{subfigure}[b]{0.4\textwidth}
	\includegraphics[scale=0.5]{graph/{output_grasp_3_k9_1_2n.csvTime}.pdf}
	\begin{center}
	$j = 9$
	\end{center}
\end{subfigure}
\begin{subfigure}[b]{0.4\textwidth}
	\includegraphics[scale=0.5]{graph/{output_grasp_3_k10_1_2n.csvTime}.pdf}
	\begin{center}
	$j = 10$
	\end{center}
\end{subfigure}
\end{figure}

Podemos ver claramente que si $j = 3$, obtenemos los mejores resultados. También se analizo los posibles valores de $k$, manteniendo $j = 5$. Se obtuvieron los siguientes resultados:

\begin{figure}[H]
\centering

\begin{subfigure}[b]{0.4\textwidth}
	\includegraphics[scale=0.5]{graph/{output_grasp_3_1_1_2n.csvTime}.pdf}
	\begin{center}
	$k = 1$
	\end{center}
\end{subfigure}
\begin{subfigure}[b]{0.4\textwidth}
	\includegraphics[scale=0.5]{graph/{output_grasp_3_2_1_2n.csvTime}.pdf}
	\begin{center}
	$k = 2$
	\end{center}
\end{subfigure}
\end{figure}

\begin{figure}[H]
\centering

\begin{subfigure}[b]{0.4\textwidth}
	\includegraphics[scale=0.5]{graph/{output_grasp_3_3_1_2n.csvTime}.pdf}
	\begin{center}
	$k = 3$
	\end{center}
\end{subfigure}
\begin{subfigure}[b]{0.4\textwidth}
	\includegraphics[scale=0.5]{graph/{output_grasp_3_4_1_2n.csvTime}.pdf}
	\begin{center}
	$k = 4$
	\end{center}
\end{subfigure}

\begin{subfigure}[b]{0.4\textwidth}
	\includegraphics[scale=0.5]{graph/{output_grasp_3_5_1_2n.csvTime}.pdf}
	\begin{center}
	$k = 5$
	\end{center}
\end{subfigure}
\begin{subfigure}[b]{0.4\textwidth}
	\includegraphics[scale=0.5]{graph/{output_grasp_3_6_1_2n.csvTime}.pdf}
	\begin{center}
	$k = 6$
	\end{center}
\end{subfigure}

\begin{subfigure}[b]{0.4\textwidth}
	\includegraphics[scale=0.5]{graph/{output_grasp_3_7_1_2n.csvTime}.pdf}
	\begin{center}
	$k = 7$
	\end{center}
\end{subfigure}
\begin{subfigure}[b]{0.4\textwidth}
	\includegraphics[scale=0.5]{graph/{output_grasp_3_8_1_2n.csvTime}.pdf}
	\begin{center}
	$k = 8$
	\end{center}
\end{subfigure}
\end{figure}

\begin{figure}[H]
\centering

\begin{subfigure}[b]{0.4\textwidth}
	\includegraphics[scale=0.5]{graph/{output_grasp_3_9_1_2n.csvTime}.pdf}
	\begin{center}
	$k = 9$
	\end{center}
\end{subfigure}
\begin{subfigure}[b]{0.4\textwidth}
	\includegraphics[scale=0.5]{graph/{output_grasp_3_10_1_2n.csvTime}.pdf}
	\begin{center}
	$k = 10$
	\end{center}
\end{subfigure}
\end{figure}

A diferencia del caso de $j$, aquí no hubo un impacto tan grande, si bien hubo diferencias, al variar el valor de $j$, no solo cambio el tiempo de convergencia, sino que ademas se estabilizo el tiempo promedio.

Ademas de los tiempos, también analizamos los tamaños de las soluciones obtenidas, para poder tomar una mejor decisión:

\begin{table}[H]
\centering
\label{my-label}
\begin{tabular}{|l|lllll|}
\hline
       & \multicolumn{1}{l|}{n = 10} & \multicolumn{1}{l|}{n = 15} & \multicolumn{1}{l|}{n = 20} & \multicolumn{1}{l|}{n = 25} & n = 30 \\ \hline
k = 1  & 3                           & 4                           & 7                           & 10                          & 11     \\ \cline{1-1}
k = 2  & 3                           & 5                           & 6                           & 10                          & 11     \\ \cline{1-1}
k = 3  & 3                           & 5                           & 7                           & 10                          & 10     \\ \cline{1-1}
k = 4  & 2                           & 5                           & 6                           & 10                          & 9      \\ \cline{1-1}
k = 5  & 3                           & 5                           & 7                           & 10                          & 10     \\ \cline{1-1}
k = 6  & 3                           & 5                           & 5                           & 10                          & 11     \\ \cline{1-1}
k = 7  & 3                           & 4                           & 6                           & 10                          & 10     \\ \cline{1-1}
k = 8  & 3                           & 5                           & 6                           & 9                           & 9      \\ \cline{1-1}
k = 9  & 3                           & 5                           & 6                           & 10                          & 9      \\ \cline{1-1}
k = 10 & 3                           & 5                           & 7                           & 9                           & 9      \\ \cline{1-1}
j = 1  & 3                           & 6                           & 7                           & 10                          & 11     \\ \cline{1-1}
j = 2  & 3                           & 5                           & 7                           & 10                          & 11     \\ \cline{1-1}
j = 3  & 3                           & 5                           & 7                           & 10                          & 11     \\ \cline{1-1}
j = 4  & 3                           & 5                           & 7                           & 10                          & 10     \\ \cline{1-1}
j = 5  & 3                           & 5                           & 7                           & 10                          & 10     \\ \cline{1-1}
j = 6  & 3                           & 5                           & 7                           & 10                          & 10     \\ \cline{1-1}
j = 7  & 3                           & 5                           & 7                           & 10                          & 10     \\ \cline{1-1}
j = 8  & 3                           & 5                           & 7                           & 10                          & 10     \\ \cline{1-1}
j = 9  & 3                           & 4                           & 7                           & 10                          & 10     \\ \cline{1-1}
j = 10 & 3                           & 4                           & 7                           & 10                          & 10     \\ \hline
\end{tabular}
\caption{Tamaños de soluciones}
\end{table}

La variación de parámetros no afecto fuertemente la calidad de las soluciones, si bien hubo casos en donde mejoro. Consideramos que si tomamos $j = 3$ y $k = 5$ pudimos conseguir una buena configuración, ya que si bien no genera la mejor solución, la diferencia no es tan grande y el tiempo de convergencia fue considerablemente menor que el del resto.

Un análisis mas exhaustivo habría sido analizar las 100 posibles combinaciones tomando valores entre 1 y 10 para $j$ y $k$. Sin embargo, los resultados obtenidos fueron buenos.