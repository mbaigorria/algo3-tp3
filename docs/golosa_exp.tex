\subsection{Experimentación}

Para la experimentación se siguió con la metodología indicada anteriormente. Los resultados fueron los siguientes.

\subsubsection{Heurística Constructiva Golosa por Grado}

Los resultados temporales obtenidos fueron los siguientes:

\begin{figure}[H]
\centering

\begin{subfigure}[h]{0.4\textwidth}
	\includegraphics[scale=0.5]{graph/{output_greedy_1_1_n.csvTime}.pdf}
	\begin{center}
	Grafos Aleatorios ($m = n$)
	\end{center}
\end{subfigure}
\begin{subfigure}[h]{0.4\textwidth}
	\includegraphics[scale=0.5]{graph/{output_greedy_1_1_2n.csvTime}.pdf}
	\begin{center}
	Grafos Aleatorios ($m = 2n$)
	\end{center}
\end{subfigure}

\begin{subfigure}[h]{0.4\textwidth}
	\includegraphics[scale=0.5]{graph/{output_greedy_1_1_n2.csvTime}.pdf}
	\begin{center}
	Grafos Aleatorios ($m = \frac{n}{2}$)
	\end{center}
\end{subfigure}
\begin{subfigure}[h]{0.4\textwidth}
	\includegraphics[scale=0.5]{graph/{output_greedy_1_3_n4.csvTime}.pdf}
	\begin{center}
	Grafos Bipartitos ($\frac{n}{4}$ nodos en la segunda componente)
	\end{center}
\end{subfigure}
\end{figure}

\begin{figure}[H]
\centering

\begin{subfigure}[h]{0.4\textwidth}
	\includegraphics[scale=0.5]{graph/{output_greedy_1_3_3n4.csvTime}.pdf}
	\begin{center}
	Grafos Bipartitos ($\frac{3n}{4}$ nodos en la segunda componente)
	\end{center}
\end{subfigure}
\begin{subfigure}[h]{0.4\textwidth}
	\includegraphics[scale=0.5]{graph/{output_greedy_1_2_n4.csvTime}.pdf}
	\begin{center}
	Grafos $d$-regulares ($d = \frac{n}{4}$)
	\end{center}
\end{subfigure}

\begin{subfigure}[h]{0.4\textwidth}
	\includegraphics[scale=0.5]{graph/{output_greedy_1_2_n2.csvTime}.pdf}
	\begin{center}
	Grafos $d$-regulares ($m = \frac{n}{2}$)
	\end{center}
\end{subfigure}
\begin{subfigure}[h]{0.4\textwidth}
	\includegraphics[scale=0.5]{graph/{output_greedy_1_2_3n4.csvTime}.pdf}
	\begin{center}
	Grafos $d$-regulares ($m = \frac{3n}{4}$)
	\end{center}
\end{subfigure}

\begin{subfigure}[h]{0.4\textwidth}
	\includegraphics[scale=0.5]{graph/{output_greedy_1_4_arbol.csvTime}.pdf}
	\begin{center}
	Arboles Binarios
	\end{center}
\end{subfigure}
\begin{subfigure}[h]{0.4\textwidth}
	\includegraphics[scale=0.5]{graph/{output_greedy_1_1_clique.csvTime}.pdf}
	\begin{center}
	Clique
	\end{center}
\end{subfigure}
\end{figure}

\newpage
Primero vamos a ver los resultados por cada familia.

\begin{itemize}
	\item Grafos aleatorios: En este caso podemos ver que la cantidad de conexiones entre nodos afecto al tiempo. De todas maneras el impacto no fue tan grande como esperábamos. En el caso $n = 120$ la diferencia entre $m = \frac{n}{4}$ y $m = 2n$ fue, en promedio, de 218 segundos.
	\item Grafos bipartitos: En este caso nos sorprendió el tiempo que tardo el algoritmo en poder encontrar solución. Consideramos que esto se debe a que en un grafo bipartito completo existen solo dos posibles cubrimientos. Otro detalle a destacar fue el aumento en tiempo que hubo mientras mas equilibradas se encontraban las dos componentes del grafo, con $\frac{3n}{4}$ nodos en la segunda componente se convergió a un resultado en un tiempo mucho mayor.
	\item Grafos $d$-regulares: Aquí a diferencia de los grafos aleatorios, al haber una diferencia mas marcada entre la cantidad de conexiones se puede ver en el gráfico que la diferencia entre $d = \frac{n}{4}$ y $d = \frac{3n}{4}$ es muy marcada, la misma siendo de varios minutos.
	\item Arboles binarios: En este caso, podemos observar que el tiempo de ejecución se comporta de forma creciente sobre $n$. Hay dos outliers que arruinan la escala del gráfico, pero podemos observar que la tendencia es como mínimo lineal. Esto tiene sentido dado que estamos agregando solo un nodo y una arista por cada aumento en $n$.
	\item Cliques: Las cliques se comportaron de manera esperada, al ser un caso fácil de resolver el algoritmo no tuvo mayores dificultades.
\end{itemize}

Para el análisis del tamaño de la solución, vamos a ver los resultados por cada familia. En el caso de los aleatorios, los resultados para estas configuraciones fueron los siguientes:

\begin{table}[H]
\centering
\label{my-label}
\begin{tabular}{|l|lll|}
\hline
        & \multicolumn{1}{l|}{m = n/2} & \multicolumn{1}{l|}{m = n} & m = 2n \\ \hline
n = 40  & 26                           & 21                         & 12     \\ \cline{1-1}
n = 60  & 38                           & 27                         & 16     \\ \cline{1-1}
n = 80  & 49                           & 33                         & 21     \\ \cline{1-1}
n = 100 & 59                           & 42                         & 24     \\ \cline{1-1}
n = 120 & 74                           & 55                         & 28     \\ \hline
\end{tabular}
\caption{Grafos aleatorios. Los números en la tabla muestran la cantidad de vértices en el conjunto dominante independiente final.}
\end{table}

Los tamaños de resultados se comportaron de manera esperada, es decir, a medida que aumento la cantidad de aristas se redujo el cardinal del conjunto solución.

Para los Grafos Bipartitos, los $d$-regulares y las cliques, el algoritmo encontró la solución optima en todos los casos. Respecto a los arboles, la solución del algoritmo siempre respeto la cota y el resultado fue el menor posible.

\subsubsection{Heurística Constructiva Golosa por Scoring}

Los resultados temporales obtenidos fueron los siguientes:

\begin{figure}[H]
\centering

\begin{subfigure}[h]{0.4\textwidth}
	\includegraphics[scale=0.5]{graph/{output_greedy_2_1_n.csvTime}.pdf}
	\begin{center}
	Grafos Aleatorios ($m = n$)
	\end{center}
\end{subfigure}
\begin{subfigure}[h]{0.4\textwidth}
	\includegraphics[scale=0.5]{graph/{output_greedy_2_1_2n.csvTime}.pdf}
	\begin{center}
	Grafos Aleatorios ($m = 2n$)
	\end{center}
\end{subfigure}

\end{figure}

\begin{figure}[H]
\centering

\begin{subfigure}[h]{0.4\textwidth}
	\includegraphics[scale=0.5]{graph/{output_greedy_2_1_n2.csvTime}.pdf}
	\begin{center}
	Grafos Aleatorios ($m = \frac{n}{2}$)
	\end{center}
\end{subfigure}
\begin{subfigure}[h]{0.4\textwidth}
	\includegraphics[scale=0.5]{graph/{output_greedy_2_3_n4.csvTime}.pdf}
	\begin{center}
	Grafos Bipartitos ($\frac{n}{4}$ nodos en la segunda componente)
	\end{center}
\end{subfigure}

\begin{subfigure}[h]{0.4\textwidth}
	\includegraphics[scale=0.5]{graph/{output_greedy_2_3_3n4.csvTime}.pdf}
	\begin{center}
	Grafos Bipartitos ($\frac{3n}{4}$ nodos en la segunda componente)
	\end{center}
\end{subfigure}
\begin{subfigure}[h]{0.4\textwidth}
	\includegraphics[scale=0.5]{graph/{output_greedy_2_2_n4.csvTime}.pdf}
	\begin{center}
	Grafos $d$-regulares ($d = \frac{n}{4}$)
	\end{center}
\end{subfigure}

\begin{subfigure}[h]{0.4\textwidth}
	\includegraphics[scale=0.5]{graph/{output_greedy_2_2_n2.csvTime}.pdf}
	\begin{center}
	Grafos $d$-regulares ($m = \frac{n}{2}$)
	\end{center}
\end{subfigure}
\begin{subfigure}[h]{0.4\textwidth}
	\includegraphics[scale=0.5]{graph/{output_greedy_2_2_3n4.csvTime}.pdf}
	\begin{center}
	Grafos $d$-regulares ($m = \frac{3n}{4}$)
	\end{center}
\end{subfigure}

\end{figure}

\begin{figure}[H]
\centering

\begin{subfigure}[b]{0.4\textwidth}
	\includegraphics[scale=0.5]{graph/{output_greedy_2_4_arbol.csvTime}.pdf}
	\begin{center}
	Arboles Binarios
	\end{center}
\end{subfigure}
\begin{subfigure}[b]{0.4\textwidth}
	\includegraphics[scale=0.5]{graph/{output_greedy_2_1_clique.csvTime}.pdf}
	\begin{center}
	Clique
	\end{center}
\end{subfigure}
\end{figure}

Los resultados obtenidos por familia no difirieron en gran medida respecto a lo obtenido con la Heurística Constructiva Golosa por Grado, con lo cual respecto al tiempo se derivan las misma conclusiones de antes.

Para el análisis del tamaño de la solución, vamos a ver los resultados por cada familia. En el caso de los aleatorios, los resultados para estas configuraciones fueron los siguiente:

\begin{table}[H]
\centering
\label{my-label}
\begin{tabular}{|l|lll|}
\hline
        & \multicolumn{1}{l|}{m = n/2} & \multicolumn{1}{l|}{m = n} & m = 2n \\ \hline
n = 40  & 32                           & 26                         & 16     \\ \cline{1-1}
n = 60  & 43                           & 33                         & 16     \\ \cline{1-1}
n = 80  & 56                           & 44                         & 30     \\ \cline{1-1}
n = 100 & 67                           & 56                         & 40     \\ \cline{1-1}
n = 120 & 74                           & 66                         & 46     \\ \hline
\end{tabular}
\caption{Grafos aleatorios. Los números en la tabla muestran la cantidad de vértices en el conjunto dominante independiente final.}
\end{table}

Aquí es donde la diferencia es mas marcada, para los mismos casos, la Heurística por Scoring dio resultados significativamente peores en el caso aleatorio. Esto también se vio reflejado en las otras familias también, particularmente en el caso de los bipartitos donde siempre se priorizo la solución mas grande.

\subsubsection{Conclusión}

En lo que respecta al tiempo de ejecución, las heurísticas no se comportaron de manera muy diferente, el tiempo fue similar. Sin embargo, recordemos que la heurística por grado utiliza un heap como estructura de datos auxiliar, por lo que para tamaños de $n$ más grandes la diferencia se notaria más. El lugar donde la diferencia fue significativa fue en el tamaño de las soluciones obtenidas, donde en prácticamente todos los casos la Heurística Constructiva por Grado dio mejor resultado, con lo cual consideramos que de las golosas, es mejor la selección por grado que por scoring.