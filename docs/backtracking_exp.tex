\subsection{Experimentacion}

Para la experimentacion de Backtracking se decidio tomar un rango de numeros mas pequeño, ya que la complejidad del algoritmo nos impedia sino obtener un resultado en tiempo razonable. Para los valres de $n$ se tomo a partir de 6 y se lo incremento en 2 hasta llegar a 24. Ademas de esto no hubo ninguna diferencia en la metodologia, ya que se probo el algoritmo para todas las familias. Primero se probo el algoritmo sin poda, el resultado obtenido fue el siguiente:

\begin{figure}[H]
\centering

\begin{subfigure}[b]{0.4\textwidth}
	\includegraphics[scale=0.6]{graph/{output_backtracking_2_1_n.csvTime}.pdf}
	\begin{center}
	Grafos Aleatorios ($m = n$)
	\end{center}
\end{subfigure}
\begin{subfigure}[b]{0.4\textwidth}
	\includegraphics[scale=0.6]{graph/{output_backtracking_2_1_2n.csvTime}.pdf}
	\begin{center}
	Grafos Aleatorios ($m = 2n$)
	\end{center}
\end{subfigure}

\begin{subfigure}[b]{0.4\textwidth}
	\includegraphics[scale=0.6]{graph/{output_backtracking_2_1_n2.csvTime}.pdf}
	\begin{center}
	Grafos Aleatorios ($m = \frac{n}{2}$)
	\end{center}
\end{subfigure}
\begin{subfigure}[b]{0.4\textwidth}
	\includegraphics[scale=0.6]{graph/{output_backtracking_2_3_n4.csvTime}.pdf}
	\begin{center}
	Grafos Bipartitos ($\frac{n}{4}$ nodos en la segunda componente)
	\end{center}
\end{subfigure}
\end{figure}

\begin{figure}[H]
\centering

\begin{subfigure}[b]{0.4\textwidth}
	\includegraphics[scale=0.6]{graph/{output_backtracking_2_3_3n4.csvTime}.pdf}
	\begin{center}
	Grafos Bipartitos ($\frac{3n}{4}$ nodos en la segunda componente)
	\end{center}
\end{subfigure}
\begin{subfigure}[b]{0.4\textwidth}
	\includegraphics[scale=0.6]{graph/{output_backtracking_2_2_n4.csvTime}.pdf}
	\begin{center}
	Grafos $d$-regulares ($d = \frac{n}{4}$)
	\end{center}
\end{subfigure}

\begin{subfigure}[b]{0.4\textwidth}
	\includegraphics[scale=0.6]{graph/{output_backtracking_2_2_n2.csvTime}.pdf}
	\begin{center}
	Grafos $d$-regulares ($m = \frac{n}{2}$)
	\end{center}
\end{subfigure}
\begin{subfigure}[b]{0.4\textwidth}
	\includegraphics[scale=0.6]{graph/{output_backtracking_2_2_3n4.csvTime}.pdf}
	\begin{center}
	Grafos $d$-regulares ($m = \frac{3n}{4}$)
	\end{center}
\end{subfigure}
\end{figure}

\begin{figure}[H]
\centering

\begin{subfigure}[b]{0.4\textwidth}
	\includegraphics[scale=0.6]{graph/{output_backtracking_2_4_arbol.csvTime}.pdf}
	\begin{center}
	Arboles Binarios
	\end{center}
\end{subfigure}
\begin{subfigure}[b]{0.4\textwidth}
	\includegraphics[scale=0.6]{graph/{output_backtracking_2_1_clique.csvTime}.pdf}
	\begin{center}
	Clique
	\end{center}
\end{subfigure}
\end{figure}

Tal como era esperable, A medida que el valor de $n$ aumento, el tiempo de ejecucion del algoritmo aumento a un ritmo acelerado. Luego se procedio a probar el algoritmo con la poda propuesta, los resultados fueron:

\begin{figure}[H]
\centering

\begin{subfigure}[b]{0.4\textwidth}
	\includegraphics[scale=0.6]{graph/{output_backtracking_1_1_n.csvTime}.pdf}
	\begin{center}
	Grafos Aleatorios ($m = n$)
	\end{center}
\end{subfigure}
\begin{subfigure}[b]{0.4\textwidth}
	\includegraphics[scale=0.6]{graph/{output_backtracking_1_1_2n.csvTime}.pdf}
	\begin{center}
	Grafos Aleatorios ($m = 2n$)
	\end{center}
\end{subfigure}

\begin{subfigure}[b]{0.4\textwidth}
	\includegraphics[scale=0.6]{graph/{output_backtracking_1_1_n2.csvTime}.pdf}
	\begin{center}
	Grafos Aleatorios ($m = \frac{n}{2}$)
	\end{center}
\end{subfigure}
\begin{subfigure}[b]{0.4\textwidth}
	\includegraphics[scale=0.6]{graph/{output_backtracking_1_3_n4.csvTime}.pdf}
	\begin{center}
	Grafos Bipartitos ($\frac{n}{4}$ nodos en la segunda componente)
	\end{center}
\end{subfigure}
\end{figure}

\begin{figure}[H]
\centering

\begin{subfigure}[b]{0.4\textwidth}
	\includegraphics[scale=0.6]{graph/{output_backtracking_1_3_3n4.csvTime}.pdf}
	\begin{center}
	Grafos Bipartitos ($\frac{3n}{4}$ nodos en la segunda componente)
	\end{center}
\end{subfigure}
\begin{subfigure}[b]{0.4\textwidth}
	\includegraphics[scale=0.6]{graph/{output_backtracking_1_2_n4.csvTime}.pdf}
	\begin{center}
	Grafos $d$-regulares ($d = \frac{n}{4}$)
	\end{center}
\end{subfigure}

\begin{subfigure}[b]{0.4\textwidth}
	\includegraphics[scale=0.6]{graph/{output_backtracking_1_2_n2.csvTime}.pdf}
	\begin{center}
	Grafos $d$-regulares ($m = \frac{n}{2}$)
	\end{center}
\end{subfigure}
\begin{subfigure}[b]{0.4\textwidth}
	\includegraphics[scale=0.6]{graph/{output_backtracking_1_2_3n4.csvTime}.pdf}
	\begin{center}
	Grafos $d$-regulares ($m = \frac{3n}{4}$)
	\end{center}
\end{subfigure}
\end{figure}

\begin{figure}[H]
\centering

\begin{subfigure}[b]{0.4\textwidth}
	\includegraphics[scale=0.6]{graph/{output_backtracking_1_4_arbol.csvTime}.pdf}
	\begin{center}
	Arboles Binarios
	\end{center}
\end{subfigure}
\begin{subfigure}[b]{0.4\textwidth}
	\includegraphics[scale=0.6]{graph/{output_backtracking_1_1_clique.csvTime}.pdf}
	\begin{center}
	Clique
	\end{center}
\end{subfigure}
\end{figure}

Aqui podemos apreciar que el resultado mejoro respecto a lo obtenido sin la poda. Sin embargo, los tiempos de ejecucion siguen siendo muy grandes.