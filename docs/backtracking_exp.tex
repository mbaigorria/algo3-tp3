\subsection{Experimentacion}

Para la experimentación de backtracking se decidió tomar un rango de números mas pequeño, ya que la complejidad del algoritmo nos impedía obtener un resultado en tiempo razonable. Para los valores de $n$ se tomo a partir de 6 y se lo incremento en 2 hasta llegar a 24. Ademas de esto no hubo ninguna diferencia en la metodología, ya que se probo el algoritmo para todas las familias. Primero se probo el algoritmo sin la poda del tamaño de la solución actual, el resultado obtenido fue el siguiente:

\begin{figure}[H]
\centering

\begin{subfigure}[b]{0.4\textwidth}
	\includegraphics[scale=0.5]{graph/{output_backtracking_2_1_n.csvTime}.pdf}
	\begin{center}
	Grafos Aleatorios ($m = n$)
	\end{center}
\end{subfigure}
\begin{subfigure}[b]{0.4\textwidth}
	\includegraphics[scale=0.5]{graph/{output_backtracking_2_1_2n.csvTime}.pdf}
	\begin{center}
	Grafos Aleatorios ($m = 2n$)
	\end{center}
\end{subfigure}

\begin{subfigure}[b]{0.4\textwidth}
	\includegraphics[scale=0.5]{graph/{output_backtracking_2_1_n2.csvTime}.pdf}
	\begin{center}
	Grafos Aleatorios ($m = \frac{n}{2}$)
	\end{center}
\end{subfigure}
\begin{subfigure}[b]{0.4\textwidth}
	\includegraphics[scale=0.5]{graph/{output_backtracking_2_3_n4.csvTime}.pdf}
	\begin{center}
	Grafos Bipartitos ($\frac{n}{4}$ nodos en la segunda componente)
	\end{center}
\end{subfigure}
\end{figure}

\begin{figure}[H]
\centering

\begin{subfigure}[b]{0.4\textwidth}
	\includegraphics[scale=0.5]{graph/{output_backtracking_2_3_3n4.csvTime}.pdf}
	\begin{center}
	Grafos Bipartitos ($\frac{3n}{4}$ nodos en la segunda componente)
	\end{center}
\end{subfigure}
\begin{subfigure}[b]{0.4\textwidth}
	\includegraphics[scale=0.5]{graph/{output_backtracking_2_2_n4.csvTime}.pdf}
	\begin{center}
	Grafos $d$-regulares ($d = \frac{n}{4}$)
	\end{center}
\end{subfigure}

\begin{subfigure}[b]{0.4\textwidth}
	\includegraphics[scale=0.5]{graph/{output_backtracking_2_2_n2.csvTime}.pdf}
	\begin{center}
	Grafos $d$-regulares ($m = \frac{n}{2}$)
	\end{center}
\end{subfigure}
\begin{subfigure}[b]{0.4\textwidth}
	\includegraphics[scale=0.5]{graph/{output_backtracking_2_2_3n4.csvTime}.pdf}
	\begin{center}
	Grafos $d$-regulares ($m = \frac{3n}{4}$)
	\end{center}
\end{subfigure}
\end{figure}

\begin{figure}[H]
\centering

\begin{subfigure}[b]{0.4\textwidth}
	\includegraphics[scale=0.5]{graph/{output_backtracking_2_4_arbol.csvTime}.pdf}
	\begin{center}
	Arboles Binarios
	\end{center}
\end{subfigure}
\begin{subfigure}[b]{0.4\textwidth}
	\includegraphics[scale=0.5]{graph/{output_backtracking_2_1_clique.csvTime}.pdf}
	\begin{center}
	Clique
	\end{center}
\end{subfigure}
\end{figure}

Como era de esperarse, a medida que el valor de $n$ aumento, el tiempo de ejecución del algoritmo aumento a un ritmo acelerado. Luego se procedió a probar el algoritmo con la poda propuesta, los resultados fueron:

\begin{figure}[H]
\centering

\begin{subfigure}[b]{0.4\textwidth}
	\includegraphics[scale=0.5]{graph/{output_backtracking_1_1_n.csvTime}.pdf}
	\begin{center}
	Grafos Aleatorios ($m = n$)
	\end{center}
\end{subfigure}
\begin{subfigure}[b]{0.4\textwidth}
	\includegraphics[scale=0.5]{graph/{output_backtracking_1_1_2n.csvTime}.pdf}
	\begin{center}
	Grafos Aleatorios ($m = 2n$)
	\end{center}
\end{subfigure}

\begin{subfigure}[b]{0.4\textwidth}
	\includegraphics[scale=0.5]{graph/{output_backtracking_1_1_n2.csvTime}.pdf}
	\begin{center}
	Grafos Aleatorios ($m = \frac{n}{2}$)
	\end{center}
\end{subfigure}
\begin{subfigure}[b]{0.4\textwidth}
	\includegraphics[scale=0.5]{graph/{output_backtracking_1_3_n4.csvTime}.pdf}
	\begin{center}
	Grafos Bipartitos ($\frac{n}{4}$ nodos en la segunda componente)
	\end{center}
\end{subfigure}

\begin{subfigure}[b]{0.4\textwidth}
	\includegraphics[scale=0.5]{graph/{output_backtracking_1_3_3n4.csvTime}.pdf}
	\begin{center}
	Grafos Bipartitos ($\frac{3n}{4}$ nodos en la segunda componente)
	\end{center}
\end{subfigure}
\begin{subfigure}[b]{0.4\textwidth}
	\includegraphics[scale=0.5]{graph/{output_backtracking_1_2_n4.csvTime}.pdf}
	\begin{center}
	Grafos $d$-regulares ($d = \frac{n}{4}$)
	\end{center}
\end{subfigure}

\end{figure}

\begin{figure}[H]
\centering

\begin{subfigure}[b]{0.4\textwidth}
	\includegraphics[scale=0.5]{graph/{output_backtracking_1_2_n2.csvTime}.pdf}
	\begin{center}
	Grafos $d$-regulares ($m = \frac{n}{2}$)
	\end{center}
\end{subfigure}
\begin{subfigure}[b]{0.4\textwidth}
	\includegraphics[scale=0.5]{graph/{output_backtracking_1_2_3n4.csvTime}.pdf}
	\begin{center}
	Grafos $d$-regulares ($m = \frac{3n}{4}$)
	\end{center}
\end{subfigure}
\end{figure}

\begin{figure}[H]
\centering

\begin{subfigure}[b]{0.4\textwidth}
	\includegraphics[scale=0.5]{graph/{output_backtracking_1_4_arbol.csvTime}.pdf}
	\begin{center}
	Arboles Binarios
	\end{center}
\end{subfigure}
\begin{subfigure}[b]{0.4\textwidth}
	\includegraphics[scale=0.5]{graph/{output_backtracking_1_1_clique.csvTime}.pdf}
	\begin{center}
	Clique
	\end{center}
\end{subfigure}
\end{figure}

Aquí podemos apreciar que el resultado mejoro respecto a lo obtenido sin la poda. Sin embargo, los tiempos de ejecución siguen siendo muy grandes. Estos resultados coinciden con el hecho de que el problema es NP-Hard.