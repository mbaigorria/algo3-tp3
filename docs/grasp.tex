\section{Metaheurística GRASP}

\subsection{Algoritmo}

GRASP (Greedy Randomized Adaptative Search Procedure) es una combinación entre una heurística golosa aleatorizada y un procedimiento de búsqueda local. La metaheurística se puede ver con el siguiente pseudocódigo:

\begin{algorithmic}
\Procedure{grasp}{G, k}
\State{G bestSolution}
\While{$!terminationCondition(G, k, bestSolution)$}
	\State{s $\gets$ randomGreedyHeuristic(G, k)}
	\State{s $\gets$ localSearch(G,s)}
	\If{$|s| < |bestSolution|$}
	\State{bestSolution $\gets$ s}
	\EndIf
\EndWhile
\EndProcedure
\end{algorithmic}

De este procedimiento surgen dos preguntas, que en realidad son cosas que debemos definir. De donde proviene la aleatoriedad de la heurística greedy? Cual es criterio de terminación que utilizaremos?

\subsection{Random Greedy Heuristic}

\subsubsection{Por cantidad}

Para agregarle una componente aleatoria a \texttt{GRASP}, se propone fabricar en cada paso de la heurística constructiva golosa una \textit{Lista Restricta de Candidatos} (RCL) y elegir aleatoriamente un candidato de esta lista. Para ello, decidimos crear la función \texttt{greedyHeapConstructiveRandomized(Node graph[], int n, int k)} que lo que hace es ir eligiendo aleatoriamente de  los $k$ vértices con mayor grado utilizando un heap como estructura auxiliar. Esto se puede ver en el siguiente pseudocódigo:

\begin{algorithmic}
\Procedure{greedyHeapConstructiveRandomized}{G,k}

\State{nodeHeap $\gets$ buildHeap(G.V)}

\While{!nodeHeap.isEmpty()}
	\State{node $\gets$ nodeHeap.topRandomPop(k)}
	\If{node.reachable == true}
		\State{continue}
	\EndIf
	\State{node.added = true}
	
	\ForAll{$adj \in node.adj$}
		\State{adj.reachable $\gets$ true}
	\EndFor
\EndWhile
\EndProcedure
\end{algorithmic}

\subsubsection{Por valor}

Al igual que en el criterio anterior, elegimos un candidato aleatorio de una lista desde un heap. Sin embargo, ahora un vértice está en la lista de candidatos si y sólo si el grado de cualquier nodo en la lista esta a una distancia menor o igual a $k$ grados del vértice de mayor grado disponible en el heap. Esto se puede ver en el siguiente pseudocódigo:

\begin{algorithmic}
\Procedure{greedyHeapConstructiveRandomized}{G,k}

\State{nodeHeap $\gets$ buildHeap(G.V)}

\While{!nodeHeap.isEmpty()}
	\State{node $\gets$ nodeHeap.topRandomPopByValue(k)}
	\If{node.reachable == true}
		\State{continue}
	\EndIf
	\State{node.added = true}
	
	\ForAll{$adj \in node.adj$}
		\State{adj.reachable $\gets$ true}
	\EndFor
\EndWhile
\EndProcedure
\end{algorithmic}

\subsection{Criterios de terminación}
\begin{enumerate}
\item No se encontró ninguna mejora en las ultimas $j$ iteraciones.
\item Se alcanzo un límite prefijado de  $j$ iteraciones.
\end{enumerate}