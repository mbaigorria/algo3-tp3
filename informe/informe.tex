\documentclass[10pt,a4paper]{article}
\usepackage[paper=a4paper, hmargin=1.5cm, bottom=1.5cm, top=3.5cm]{geometry}

\usepackage[utf8]{inputenc}
\usepackage[spanish]{babel}

\usepackage{mathtools}
\usepackage{amsmath}
\usepackage{amsfonts}
\usepackage{amssymb}

\usepackage{xcolor}
\usepackage{listingsutf8}
\usepackage{booktabs}
\usepackage{hyperref}

\usepackage{caption}
\usepackage{subcaption}

\usepackage{algpseudocode}
\usepackage{ifthen}
\usepackage{enumitem}

\usepackage{graphicx}
\usepackage{tikz}

\DeclarePairedDelimiter{\ceil}{\lceil}{\rceil}

%\let\NombreFuncion=\textsc
%\let\TipoVariable=\texttt

%\newcommand{\TipoFuncion}[3]{%
  %\NombreFuncion{#1}(#2) \ifx#3\empty\else $\to$ \res\,: \TipoVariable{#3}\fi%
%}

% set the default code style
\lstset{
    frame=tb, % draw a frame at the top and bottom of the code block
    tabsize=4, % tab space width
    showstringspaces=false, % don't mark spaces in strings
    numbers=left, % display line numbers on the left
    commentstyle=\color{green}, % comment color
    keywordstyle=\color{blue}, % keyword color
    stringstyle=\color{red} % string color
}

% mathy stuff
\newtheorem{theorem}{Theorem}[section]
\newtheorem{lemma}[theorem]{Lemma}
\newtheorem{proposition}[theorem]{Proposición}
\newtheorem{corollary}[theorem]{Corollary}

\newenvironment{proof}[1][Demostración]{\begin{trivlist}
\item[\hskip \labelsep {\bfseries #1}]}{\end{trivlist}}
\newenvironment{definition}[1][Definición]{\begin{trivlist}
\item[\hskip \labelsep {\bfseries #1}]}{\end{trivlist}}
\newenvironment{example}[1][Example]{\begin{trivlist}
\item[\hskip \labelsep {\bfseries #1}]}{\end{trivlist}}
\newenvironment{remark}[1][Remark]{\begin{trivlist}
\item[\hskip \labelsep {\bfseries #1}]}{\end{trivlist}}

\newcommand{\qed}{\nobreak \ifvmode \relax \else
      \ifdim\lastskip<1.5em \hskip-\lastskip
      \hskip1.5em plus0em minus0.5em \fi \nobreak
      \vrule height0.75em width0.5em depth0.25em\fi}

\title{Algoritmos y Estructuras de Datos III \\ TP3}

\newcommand{\order}[1]{$\mathcal{O}(#1)$}

\begin{document}

%% cover page

\maketitle

\bigskip

\begin{table}[h]
\centering
\begin{tabular}{|l l l|}
\hline
Integrante       & \multicolumn{1}{c}{LU}     & Correo electrónico        \\ \hline
Martin Baigorria & \multicolumn{1}{c}{575/14} & martinbaigorria@gmail.com \\ 
Federico Beuter & 827/13                      & federicobeuter@gmail.com \\
Juan Rinaudo & 864/13                      & jangamesdev@gmail.com \\ 
Mauro Cherubini & 835/13                      & cheru.mf@gmail.com \\ \hline
\end{tabular}
\end{table}

\vfill

\begin{center}
\textbf{Reservado para la cátedra}
\end{center}
\begin{table}[h]
\centering
\begin{tabular}{|l|l|l|}
\hline
Instancia       & Docente & Nota \\ \hline
Primera entrega &         &      \\ \hline
Segunda entrega &         &      \\ \hline
\end{tabular}
\end{table}

\newpage
\tableofcontents
\newpage

% end cover page

\section{Introducción}

\subsection{Definiciones}

Antes de enunciar el problema a resolver en este trabajo practico, es necesario definir algunos conceptos.

Sea $G = (V,E)$ un grafo simple:
\begin{definition}
Un conjunto $I \subseteq V$ es un \textit{conjunto independiente} de $G$ si no existe ningún eje de $E$ entre los vértices de $I$. Es decir, los ejes de $I$ no están conectados por las aristas de $G$.
\end{definition}

\begin{definition}
Un conjunto $D \subseteq V$ es un \textit{conjunto dominante} de G si todo vértice de $G$ esta en $D$ o bien tiene al menos un vecino que esta en $D$.
\end{definition}

\begin{definition}
Un conjunto \textit{conjunto independiente dominante} de $G$ es un conjunto independiente que a su vez es dominante del grafo G. Desde un conjunto independiente dominante se puede acceder a cualquier vértice del grafo $G$ con solo recorrer una arista desde uno de sus vértices.
\end{definition}

\begin{definition}
Un \textit{Conjunto Independiente Dominante Mínimo} (CIDM) es el conjunto independiente dominante de $G$ de mínima cardinalidad.
\end{definition}

Cada definición debería ser acompañada con un gráfico. Por ejemplo, podemos mostrar dos conjuntos independientes y dominantes del mismo grafo, donde uno ese el CIDM.

\subsection{Introducción}
En 1979, Garey y Johnson probaron que el problema de encontrar el CIDM de un grafo es un problema NP-Hard\footnote{M.R. Garey, D.S. Johnson, Computers and Intractability: A Guide to the Theory of NP-Completeness, Freeman and Company, San Francisco (1979).}.
El objetivo del trabajo es utilizar diferentes técnicas algorítmicas para resolver este problema. En un principio diseñaremos e implementaremos un algoritmo exacto para el mismo. Dada la complejidad del problema, luego propondremos diferentes algoritmos heurísticos para llegar a una solución que sea lo suficientemente buena a fines prácticos en un tiempo razonable.

Si recordamos el problema 3 del TP1, podemos ver claramente que el mismo es un caso particular del problema del conjunto independiente dominante optimo. Esto se debe a que el problema de los caballos imponía cierta estructura sobre el grafo en el que se efectuaba la búsqueda. El grafo en si no era completo, dado que cada casilla era representada por un nodo, y un caballo no podía acceder a los nodos adyacentes. El movimiento de los caballos se modelaba con aristas entre nodos. En cambio, el problema de encontrar el CIDM se aplica a cualquier tipo de grafo. Dado que el problema de los caballos era computacionalmente costoso, podemos inferir, como ya lo confirma la literatura, que este problema se resolverá en tiempo no polinomial.

\subsection{Maximalidad y dominancia}

Las siguientes proposiciones serán útiles a lo largo del trabajo:

\begin{proposition}
Sea M un conjunto independiente maximal de G. $\forall v \in G.V$, si $v \notin M \implies \exists u \in M$ tal que $u$ es adyacente a $v$. 
\end{proposition}

\begin{proof}
Por absurdo. Sea M un conjunto independiente maximal y $v \notin G.V$. $\not\exists u \in M$ tal que $u$ es adyacente a $v$. Por lo tanto, puedo agregar $v$ a $M$ y el conjunto va a seguir siendo independiente. Esto es absurdo, dado que el conjunto era maximal.
\end{proof}

\begin{proposition}
Dado $G(V,E)$, todo conjunto independiente maximal es un conjunto independiente dominante.
\end{proposition}

\begin{proof}
Sea $M$ un conjunto independiente maximal. Por la propiedad anterior, si $v \notin M \implies \exists u \in M$ tal que $u$ es adyacente a $v$. Por lo tanto, si $v \notin M$ entonces tiene algún vecino que esta en $M$. Esto significa que $M$ es dominante.
\end{proof}

\newpage

\subsection{Modelado}
Muchos problemas se pueden modelar con grafos y se pueden resolver mediante la búsqueda del conjunto independiente dominante mínimo.

\subsubsection{Planificador Urbano}

Supongamos que un planificador urbano esta diseñando una ciudad con muchos barrios. Con el objetivo de proveer un buen sistema de salud para los habitantes, el planificador determina que cada barrio debe tener que cruzar a lo sumo un barrio para acceder a un hospital publico. Aquí podemos modelar a cada barrio con un vértice, y representar la adyacencia entre barrios con una arista. Al obtener el CIDM, obtenemos la ubicación y la mínima cantidad de hospitales públicos necesarios para cumplir con los objetivos del planificador.


\end{document}