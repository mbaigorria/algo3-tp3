\documentclass[10pt,a4paper]{article}
\usepackage[paper=a4paper, hmargin=1.5cm, bottom=1.5cm, top=3cm]{geometry}

\usepackage[utf8]{inputenc}
\usepackage[spanish]{babel}

\usepackage{mathtools}
\usepackage{amsmath}
\usepackage{amsfonts}
\usepackage{amssymb}

\usepackage{xcolor}
\usepackage{listingsutf8}
\usepackage{booktabs}
\usepackage{hyperref}

\usepackage{caption}
\usepackage{subcaption}

\usepackage{algorithm}
\usepackage[noend]{algpseudocode}

\usepackage{graphicx}
\usepackage{tikz}

\DeclarePairedDelimiter{\ceil}{\lceil}{\rceil}

%\let\NombreFuncion=\textsc
%\let\TipoVariable=\texttt

%\newcommand{\TipoFuncion}[3]{%
  %\NombreFuncion{#1}(#2) \ifx#3\empty\else $\to$ \res\,: \TipoVariable{#3}\fi%
%}

% set the default code style
\lstset{
    frame=tb, % draw a frame at the top and bottom of the code block
    tabsize=4, % tab space width
    showstringspaces=false, % don't mark spaces in strings
    numbers=left, % display line numbers on the left
    commentstyle=\color{green}, % comment color
    keywordstyle=\color{blue}, % keyword color
    stringstyle=\color{red} % string color
}

% mathy stuff
\newtheorem{theorem}{Theorem}[section]
\newtheorem{lemma}[theorem]{Lemma}
\newtheorem{proposition}[theorem]{Proposición}
\newtheorem{corollary}[theorem]{Corollary}

\newenvironment{proof}[1][Demostración]{\begin{trivlist}
\item[\hskip \labelsep {\bfseries #1}]}{\end{trivlist}}
\newenvironment{definition}[1][Definición]{\begin{trivlist}
\item[\hskip \labelsep {\bfseries #1}]}{\end{trivlist}}
\newenvironment{example}[1][Example]{\begin{trivlist}
\item[\hskip \labelsep {\bfseries #1}]}{\end{trivlist}}
\newenvironment{remark}[1][Remark]{\begin{trivlist}
\item[\hskip \labelsep {\bfseries #1}]}{\end{trivlist}}

\newcommand{\qed}{\nobreak \ifvmode \relax \else
      \ifdim\lastskip<1.5em \hskip-\lastskip
      \hskip1.5em plus0em minus0.5em \fi \nobreak
      \vrule height0.75em width0.5em depth0.25em\fi}

\title{Algoritmos y Estructuras de Datos III \\ TP3}

\newcommand{\order}[1]{$\mathcal{O}(#1)$}

\begin{document}

%% cover page

\maketitle

\bigskip

\begin{table}[h]
\centering
\begin{tabular}{|l l l|}
\hline
Integrante       & \multicolumn{1}{c}{LU}     & Correo electrónico        \\ \hline
Martin Baigorria & \multicolumn{1}{c}{575/14} & martinbaigorria@gmail.com \\ 
Federico Beuter & 827/13                      & federicobeuter@gmail.com \\
Juan Rinaudo & 864/13                      & jangamesdev@gmail.com \\ 
Mauro Cherubini & 835/13                      & cheru.mf@gmail.com \\ \hline
\end{tabular}
\end{table}

\vfill

\begin{center}
\textbf{Reservado para la cátedra}
\end{center}
\begin{table}[h]
\centering
\begin{tabular}{|l|l|l|}
\hline
Instancia       & Docente & Nota \\ \hline
Primera entrega &         &      \\ \hline
Segunda entrega &         &      \\ \hline
\end{tabular}
\end{table}

\newpage
\tableofcontents
\newpage

% end cover page

\section{Introducción}

\subsection{Definiciones}

Antes de enunciar el problema a resolver en este trabajo practico, es necesario definir algunos conceptos.

Sea $G = (V,E)$ un grafo simple:
\begin{definition}
Un conjunto $I \subseteq V$ es un \textit{conjunto independiente} de $G$ si no existe ningún eje de $E$ entre los vértices de $I$. Es decir, los ejes de $I$ no están conectados por las aristas de $G$.
\end{definition}

\begin{definition}
Un conjunto $D \subseteq V$ es un \textit{conjunto dominante} de G si todo vértice de $G$ esta en $D$ o bien tiene al menos un vecino que esta en $D$.
\end{definition}

\begin{definition}
Un conjunto \textit{conjunto independiente dominante} de $G$ es un conjunto independiente que a su vez es dominante del grafo G. Desde un conjunto independiente dominante se puede acceder a cualquier vértice del grafo $G$ con solo recorrer una arista desde uno de sus vértices.
\end{definition}

\begin{definition}
Un \textit{Conjunto Independiente Dominante Mínimo} (CIDM) es el conjunto independiente dominante de $G$ de mínima cardinalidad.
\end{definition}

\subsection{Introducción}
En 1979, Garey y Johnson probaron que el problema de encontrar el CIDM de un grafo es un problema NP-Hard\footnote{M.R. Garey, D.S. Johnson, Computers and Intractability: A Guide to the Theory of NP-Completeness, Freeman and Company, San Francisco (1979).}.
El objetivo del trabajo es utilizar diferentes técnicas algorítmicas para resolver este problema. En un principio diseñaremos e implementaremos un algoritmo exacto para el mismo. Dada la complejidad del problema, luego propondremos diferentes algoritmos heurísticos para llegar a una solución que sea lo suficientemente buena a fines prácticos en un tiempo razonable.

Si recordamos el problema 3 del TP1, podemos ver claramente que el mismo es un caso particular del problema del conjunto dominante mínimo. En este problema se imponía cierta estructura sobre el grafo en el que se efectuaba la búsqueda. El grafo en si no era completo, dado que cada casilla era representada por un nodo, y un caballo no podía acceder a los nodos adyacentes. El movimiento de los caballos se modelaba con aristas entre nodos. Este no es un caso del CIDM dado que la solución optima al problema (cubrir el tablaro con la menor cantidad de caballos posibles) no necesariamente era independiente. Por lo tanto, al buscar la solución estaríamos buscando el CDM del grafo.

\subsection{Maximalidad y dominancia}

Las siguientes proposiciones serán útiles a lo largo del trabajo:

\begin{proposition}
Sea M un conjunto independiente maximal de G. $\forall v \in G.V$, si $v \notin M \implies \exists u \in M$ tal que $u$ es adyacente a $v$. 
\end{proposition}

\begin{proof}
Por absurdo. Sea M un conjunto independiente maximal y $v \notin G.V$. $\not\exists u \in M$ tal que $u$ es adyacente a $v$. Por lo tanto, puedo agregar $v$ a $M$ y el conjunto va a seguir siendo independiente. Esto es absurdo, dado que el conjunto era maximal.
\end{proof}

\begin{proposition}
Dado $G(V,E)$, todo conjunto independiente maximal es un conjunto independiente dominante.
\end{proposition}

\begin{proof}
Sea $M$ un conjunto independiente maximal. Dado $v \in G.V$, por la propiedad anterior, si $v \notin M \implies \exists u \in M$ tal que $u$ es adyacente a $v$. Por lo tanto, si $v \notin M$ entonces $v$ tiene algún vecino que esta en $M$. Esto significa que $M$ es dominante.
\end{proof}

\newpage

\subsection{Modelado}
Muchos problemas se pueden modelar con grafos y se pueden resolver mediante la búsqueda del conjunto independiente dominante mínimo.

\subsubsection{Planificador Urbano}

Supongamos que un planificador urbano esta diseñando una ciudad con muchos barrios. Con el objetivo de proveer un buen sistema de salud para los habitantes, el planificador determina que cada barrio debe tener que cruzar a lo sumo un barrio para acceder a un hospital publico. Aquí podemos modelar a cada barrio con un vértice, y representar la adyacencia entre barrios con una arista. Al obtener el CIDM, obtenemos la ubicación y la mínima cantidad de hospitales públicos necesarios para cumplir con los objetivos del planificador.
\newpage

\section{Algoritmo Exacto}

\subsection{Algoritmo}
Utilizando backtracking, recorremos todas los conjuntos dominantes independientes y luego seleccionamos el de menor cardinalidad.
Representamos al grafo con un arreglo $graph[n]$ de nodos. Cada nodo tiene los siguientes atributos:

\begin{enumerate}
	\item adj: Lista de nodos adyacentes al nodo actual.
	\item degree: Grado del nodo actual.
	\item added: Bool que indica si el nodo ha sido agregado al conjunto que representa el cubrimiento.
	\item reachable: Bool que indica si el nodo actual puede ser alcanzado desde un nodo perteneciente al cubrimiento.
\end{enumerate}

Comenzamos definiendo la función $backtracking$, que lo que hace es tomar un nodo del grafo, y luego considera los casos en los que el nodo pertenece o no a un posible cubrimiento. En caso de agregar el nodo al cubrimiento, todos los nodos adyacentes al mismo son ignorados en futuras llamadas recursivas. Si consideráramos los nodos adyacentes, romperíamos la independencia de los  cubrimientos y ademas no solo incrementaría la complejidad del código sino que también el tiempo de ejecución del mismo.

\subsection{Podas y estrategias}

Para poder resolver el problema lo mas rápido posible, en primer lugar buscamos una forma rápida de verificar si un conjunto solución encontrado es independiente. En vez de tener que verificarlo, decidimos forzar la independencia por construcción. Esto se logro evitando los nodos adyacentes a los que ya agrego el algoritmo al potencial conjunto solución. De esta forma mantenemos la independencia del conjunto y evitamos tener que agregar innecesariamente muchos nodos. 

Otro problema importante es verificar si los nodos seleccionados forman un cubrimiento. Esto lo resolvimos simplemente haciendo que la función backtracking lleve la cuenta del total de nodos alcanzables por el cubrimiento. Si ese numero es igual al numero total de nodos, significa que llegamos a un cubrimiento. De esta manera evitamos funciones auxiliares que tengan que verificar si los nodos seleccionados hasta ahora forman un cubrimiento, y a su vez sabemos que por construcción el mismo es independiente.

Ademas, antes de comenzar la búsqueda agregamos todos los vértices de $d(v) = 0$ al conjunto solución final. Esto se debe a que estos vértices necesariamente estarán en la solución. Es muy simple probar esto, dado que si no lo estuvieran, algún vértice adyacente debería estar en el conjunto para que lo cubra. Sin embargo, tal vértice no existe.

Una poda muy común que también hemos implementado es la de la solución local actual. Dada una solución posible (que aun no sabemos si es la mínima), si en el estado actual del algoritmo se esta considerando un numero de vértices que no le puede ganar a esta solución, ignoramos esa rama del árbol de estados posibles.

\subsection{Complejidad}

\subsection{Complejidad Espacial}
Para la representación del grafo, utilizamos un arreglo de nodos. Cada nodo tiene una lista de adyacencia. Por lo tanto, la complejidad espacial de nuestro algoritmo es de \order{n + 2m}, donde $n$ es la cantidad total de vértices y $m$ la cantidad total de aristas.

\subsection{Complejidad Temporal}
Nuestro algoritmo, sin considerar las podas, recorre cada conjunto independiente dominante una vez. Cada vez que encuentra uno, lo guarda en una estructura auxiliar en \order{n}. Si todos los nodos tienen grado 0, son agregados automaticamente, y el algoritmo resuelve el problema en $n$ iteraciones.
En el peor de los casos, el algoritmo recorre todos los conjuntos independientes y dominantes, comenzando con el de mayor cardinalidad. Cada vez que lo encuentra, actualiza la estructura donde guardamos la solución. Para que esto suceda, en realidad todos los conjuntos dominantes deben tener diferente cardinalidad, cosa que en general no sucede. Como todo conjunto tiene $2^n$ subconjuntos, utilizaremos esto para acotar la cantidad de veces que actualiza la solución local. Seguramente hay una cota teórica mucho mejor.

%Sin embargo, asumir que si sirve para acotar la complejidad. Moon \& Moser probaron que un grafo de $n$ vértices tiene a lo sumo $3^{n/3}$ conjuntos independientes y dominantes 
%\footnote{Moon, J. W.; Moser, Leo (1965), 'On cliques in graphs', Israel Journal of Mathematics 3 (1): 23–28, doi:10.1007/BF02760024}. Por lo tanto, una cota no muy buena para la actualización de soluciones locales es \order{n \times 3^{n/3}}.z

Por otro lado, recorremos cada vértice y sus aristas adyacentes una vez por iteración. Aunque por construcción forzamos la independencia de los vértices, para poder acotar la complejidad supongamos que no ignora ninguna ramificación. Por lo tanto, la cantidad de nodos recorridos esta acotada por $2^n$. Esto significa que el algoritmo pertenece a \order{n \times 2^n}.
\newpage
%\subsection{Experimentacion}

Para la experimentacion de backtracking se tomo de 6 a 40 nodos, y para cada familia de grafos se tomaron diferentes parametros para la misma cantidad de nodos. Los resultados fueron los siguientes:\\

\begin{figure}[ht]
\centering
	\includegraphics[scale=0.45]{images/graph_ej1/output_backtracking_1_n2}
	\includegraphics[scale=0.45]{images/graph_ej1/output_backtracking_1_n}
	\includegraphics[scale=0.45]{images/graph_ej1/output_backtracking_1_2n}
\end{figure}
\newpage
\subsection{Codigo}
\lstinputlisting[language=C++, breaklines=true]{../src/backtracking/backtracking.cpp}
\newpage

\section{Heurística Constructiva Golosa}

\subsection{Algoritmo}

Para poder armar un algoritmo para el problema del CIMD, en primer lugar hay que buscar un buen criterio para seleccionar que nodos pertenecerán al cubrimiento, dado los nodos que ya han sido agregados. Decidimos utilizar como criterio el numero de nodos adyacentes efectivos a los que cada nodo puede acceder. Definimos a un nodo adyacente efectivo (score) como un nodo que es adyacente y a su vez no puede ser accedido por otros nodos que ya pertenecen al cubrimiento. De esta forma, este criterio también nos garantiza la independencia del conjunto, dado que si tomamos dos nodos de la solución, por construcción no pueden ser adyacentes.

Cada nodo va a tener como atributos su score, un flag que indica si ha sido agregado y otro que indica si es alcanzable por el cubrimiento actual.

El algoritmo va a iterar un arreglo de nodos $n^2$ veces. Cada vez que busquemos un nodo para agregar al conjunto, los iteraremos todos para buscar el de maximo score. Luego, al identificarlo, actualizaremos los scores de los nodos adyacentes a los adyacentes del mismo.

\subsection{Complejidad}

El algoritmo recorre cada nodo del arreglo $n$ veces. A su vez, actualizar los scores al identificar un máximo se hace $m$ veces. Por lo tanto, el algoritmo tiene orden \order{n^2 + m}.

Notar que la forma en que buscamos el máximo es sumamente ineficiente. Esto se debe a que si utilizamos sort, luego es bastante difícil encontrar el nodo al que le debemos actualizar su respectivo score. A su vez, dado que en cada iteracion actualizamos el score, mantener el orden es sumamente costoso. Es muy posible que exista una estructura de datos mucho mas eficiente para resolver este problema.

Podríamos resolver el problema en \order{n \times log(n)} simplemente ignorando la actualización de los scores, desencolando de un heap $n$ veces. Sin embargo, este criterio es a simple vista inferior que el de actualización de scores. Aquí hay un tradeoff entre hacer el mejor pick y la complejidad temporal del algoritmo.

\subsection{Efectividad de la heurística}

Nuestra heuristica no siempre devuelve la solucion optima. Considerar los siguientes ejemplos::

\begin{figure}[ht]
\centering
\begin{subfigure}[b]{0.4\textwidth}
	\includegraphics[scale=0.6]{images/greedy_fail.png}
	\caption{Greedy (5 nodos)}
\end{subfigure}
\begin{subfigure}[b]{0.4\textwidth}
	\includegraphics[scale=0.6]{images/greedy_best.png}
	\caption{Optimo (3 nodos)}
\end{subfigure}
\end{figure}

\begin{figure}[ht]
\centering
\begin{subfigure}[b]{0.4\textwidth}
	\includegraphics[scale=0.6]{images/greedy_fail2.png}
	\caption{Greedy (9 nodos)}
\end{subfigure}
\begin{subfigure}[b]{0.4\textwidth}
	\includegraphics[scale=0.6]{images/greedy_best2.png}
	\caption{Optimo (4 nodos)}
\end{subfigure}
\caption{Ejemplos de nuestra heuristica comparado con el optimo.}
\end{figure}
\newpage
%\subsection{Experimentación}

Para la experimentación se siguió con la metodología indicada anteriormente. Los resultados fueron los siguientes.

\subsubsection{Heuristica Constructiva Golosa por Grado}

Los resultados temporales obtenidos fueron los siguientes:

\begin{figure}[ht]
\centering

\begin{subfigure}[b]{0.4\textwidth}
	\includegraphics[scale=0.6]{graph/{output_greedy_1_1_n.csvTime}.pdf}
	\begin{center}
	Grafos Aleatorios ($m = n$)
	\end{center}
\end{subfigure}
\begin{subfigure}[b]{0.4\textwidth}
	\includegraphics[scale=0.6]{graph/{output_greedy_1_1_2n.csvTime}.pdf}
	\begin{center}
	Grafos Aleatorios ($m = 2n$)
	\end{center}
\end{subfigure}

\begin{subfigure}[b]{0.4\textwidth}
	\includegraphics[scale=0.6]{graph/{output_greedy_1_1_n2.csvTime}.pdf}
	\begin{center}
	Grafos Aleatorios ($m = \frac{n}{2}$)
	\end{center}
\end{subfigure}
\begin{subfigure}[b]{0.4\textwidth}
	\includegraphics[scale=0.6]{graph/{output_greedy_1_3_n4.csvTime}.pdf}
	\begin{center}
	Grafos Bipartitos ($\frac{n}{4}$ nodos en la segunda componente)
	\end{center}
\end{subfigure}
\end{figure}

\begin{figure}[ht]
\centering

\begin{subfigure}[b]{0.4\textwidth}
	\includegraphics[scale=0.6]{graph/{output_greedy_1_3_3n4.csvTime}.pdf}
	\begin{center}
	Grafos Bipartitos ($\frac{3n}{4}$ nodos en la segunda componente)
	\end{center}
\end{subfigure}
\begin{subfigure}[b]{0.4\textwidth}
	\includegraphics[scale=0.6]{graph/{output_greedy_1_2_n4.csvTime}.pdf}
	\begin{center}
	Grafos $d$-regulares ($d = \frac{n}{4}$)
	\end{center}
\end{subfigure}

\begin{subfigure}[b]{0.4\textwidth}
	\includegraphics[scale=0.6]{graph/{output_greedy_1_2_n2.csvTime}.pdf}
	\begin{center}
	Grafos $d$-regulares ($m = \frac{n}{2}$)
	\end{center}
\end{subfigure}
\begin{subfigure}[b]{0.4\textwidth}
	\includegraphics[scale=0.6]{graph/{output_greedy_1_2_3n4.csvTime}.pdf}
	\begin{center}
	Grafos $d$-regulares ($m = \frac{3n}{4}$)
	\end{center}
\end{subfigure}
\end{figure}

\newpage
\begin{figure}
\centering

\begin{subfigure}[b]{0.4\textwidth}
	\includegraphics[scale=0.6]{graph/{output_greedy_1_4_arbol.csvTime}.pdf}
	\begin{center}
	Arboles Binarios
	\end{center}
\end{subfigure}
\begin{subfigure}[b]{0.4\textwidth}
	\includegraphics[scale=0.6]{graph/{output_greedy_1_1_clique.csvTime}.pdf}
	\begin{center}
	Clique
	\end{center}
\end{subfigure}
\begin{center}
Greedy Heap
\end{center}
\end{figure}

\newpage
Primero vamos a ver los resultados por cada familia.

\begin{itemize}
	\item Grafos Aleatorios: En este caso podemos ver que la cantidad de conexiones entre nodos afecto al tiempo, igualmente el impacto no fue tan grande como esperábamos, en el caso $n = 120$ la diferencia entre $m = \frac{n}{4}$ y $m = 2n$ fue, en promedio, de 218 segundos.
	\item Grafos Bipartitos: En este caso nos sorprendió el tiempo que tardo el algoritmo en poder encontrar solución, consideramos que esto se debe a que en un grafo bipartito completo existen solo dos posibles cubrimientos. Otro detalle a destacar, fue el aumento en tiempo que hubo mientras mas equilibradas se encontraban las dos componentes del grafo, con $\frac{3n}{4}$ nodos en la segunda componente se convergió a un resultado en un tiempo mucho mayor.
	\item Grafos $d$-regulares: Aquí a diferencia de los grafos aleatorios, al haber una diferencia mas marcada entre la cantidad de conexiones se puede ver en el gráfico que la diferencia entre $d = \frac{n}{4}$ y $d = \frac{3n}{4}$ es muy marcada, la misma siendo de varios minutos.
	\item Arboles binarios: En este caso lamentablemente no es posible hacer un análisis detallado, ya que los resultados no fueron regulares. Sin embargo, podemos destacar, que los resultados se obtuvieron en un tiempo razonable.
	\item Cliques: Las cliques se comportaron de manera esperada, al ser un caso facil de resolver el algoritmo no tuvo mayores dificultades.
\end{itemize}

Para el análisis del tamaño de la solución, vamos a ver los resultados por cada familia. En el caso de los aleatorios, los resultados para estas configuraciones fueron los siguiente:

\begin{table}[]
\centering
\caption{Grafos aleatorios}
\label{my-label}
\begin{tabular}{|l|lll|}
\hline
        & \multicolumn{1}{l|}{m = n/2} & \multicolumn{1}{l|}{m = n} & m = 2n \\ \hline
n = 40  & 26                           & 21                         & 12     \\ \cline{1-1}
n = 60  & 38                           & 27                         & 16     \\ \cline{1-1}
n = 80  & 49                           & 33                         & 21     \\ \cline{1-1}
n = 100 & 59                           & 42                         & 24     \\ \cline{1-1}
n = 120 & 74                           & 55                         & 28     \\ \hline
\end{tabular}
\end{table}

Los tamaños de resultados se comportaron de manera esperada, es decir, a medida que avanzo la cantidad de conexión se redujo el tamaño de solución.

Para los Grafos Bipartitos, los $d$-regulares y las cliques, el algoritmo encontró la solución optima en todos los casos. Respecto a los arboles, la solución del algoritmo siempre respeto la cota y el resultado fue el menor posible.

\subsubsection{Heurística Constructiva Golosa por Scoring}

Los resultados temporales obtenidos fueron los siguientes:

\begin{figure}[H]
\centering

\begin{subfigure}[h]{0.4\textwidth}
	\includegraphics[scale=0.6]{graph/{output_greedy_2_1_n.csvTime}.pdf}
	\begin{center}
	Grafos Aleatorios ($m = n$)
	\end{center}
\end{subfigure}
\begin{subfigure}[h]{0.4\textwidth}
	\includegraphics[scale=0.6]{graph/{output_greedy_2_1_2n.csvTime}.pdf}
	\begin{center}
	Grafos Aleatorios ($m = 2n$)
	\end{center}
\end{subfigure}

\begin{subfigure}[h]{0.4\textwidth}
	\includegraphics[scale=0.6]{graph/{output_greedy_2_1_n2.csvTime}.pdf}
	\begin{center}
	Grafos Aleatorios ($m = \frac{n}{2}$)
	\end{center}
\end{subfigure}
\begin{subfigure}[h]{0.4\textwidth}
	\includegraphics[scale=0.6]{graph/{output_greedy_2_3_n4.csvTime}.pdf}
	\begin{center}
	Grafos Bipartitos ($\frac{n}{4}$ nodos en la segunda componente)
	\end{center}
\end{subfigure}
\end{figure}

\begin{figure}[H]
\centering

\begin{subfigure}[h]{0.4\textwidth}
	\includegraphics[scale=0.6]{graph/{output_greedy_2_3_3n4.csvTime}.pdf}
	\begin{center}
	Grafos Bipartitos ($\frac{3n}{4}$ nodos en la segunda componente)
	\end{center}
\end{subfigure}
\begin{subfigure}[h]{0.4\textwidth}
	\includegraphics[scale=0.6]{graph/{output_greedy_2_2_n4.csvTime}.pdf}
	\begin{center}
	Grafos $d$-regulares ($d = \frac{n}{4}$)
	\end{center}
\end{subfigure}

\begin{subfigure}[h]{0.4\textwidth}
	\includegraphics[scale=0.6]{graph/{output_greedy_2_2_n2.csvTime}.pdf}
	\begin{center}
	Grafos $d$-regulares ($m = \frac{n}{2}$)
	\end{center}
\end{subfigure}
\begin{subfigure}[h]{0.4\textwidth}
	\includegraphics[scale=0.6]{graph/{output_greedy_2_2_3n4.csvTime}.pdf}
	\begin{center}
	Grafos $d$-regulares ($m = \frac{3n}{4}$)
	\end{center}
\end{subfigure}
\end{figure}

\newpage
\begin{figure}
\centering

\begin{subfigure}[b]{0.4\textwidth}
	\includegraphics[scale=0.6]{graph/{output_greedy_2_4_arbol.csvTime}.pdf}
	\begin{center}
	Arboles Binarios
	\end{center}
\end{subfigure}
\begin{subfigure}[b]{0.4\textwidth}
	\includegraphics[scale=0.6]{graph/{output_greedy_2_1_clique.csvTime}.pdf}
	\begin{center}
	Clique
	\end{center}
\end{subfigure}
\begin{center}
Greedy Heap
\end{center}
\end{figure}

\newpage
Primero vamos a ver los resultados por cada familia.

Los resultados obtenidos por familia no difirieron en gran medida respecto a lo obtenido con la Heurística Constructiva Golosa por Grado, con lo cual respecto al tiempo se derivan las misma conclusiones de antes.

Para el análisis del tamaño de la solución, vamos a ver los resultados por cada familia. En el caso de los aleatorios, los resultados para estas configuraciones fueron los siguiente:

\begin{table}[]
\centering
\caption{My caption}
\label{my-label}
\begin{tabular}{|l|lll|}
\hline
        & \multicolumn{1}{l|}{m = n/2} & \multicolumn{1}{l|}{m = n} & m = 2n \\ \hline
n = 40  & 32                           & 26                         & 16     \\ \cline{1-1}
n = 60  & 43                           & 33                         & 16     \\ \cline{1-1}
n = 80  & 56                           & 44                         & 30     \\ \cline{1-1}
n = 100 & 67                           & 56                         & 40     \\ \cline{1-1}
n = 120 & 74                           & 66                         & 46     \\ \hline
\end{tabular}
\end{table}

Aquí es donde la diferencia es mas marcada, para los mismos casos, la Heurística por Scoring dio resultados significativamente peores en el caso aleatorio. Esto también se vio reflejado en las otras familias también, particularmente en el caso de los bipartitos donde siempre se priorizo la solución mas grande.

\subsubsection{Conclusión}

En lo que respecta al tiempo de ejecución, las heurísticas no se comportaron de manera muy diferente, el tiempo fue similar. Sin embargo, el lugar donde se noto la diferencia fue en el tamaño de las soluciones obtenidas, en prácticamente todos los casos, la Heurística Constructiva por Grado dio mejor resultado, con lo cual consideramos que de las golosas, es mejor la selección por grado que por scoring.
\newpage
\subsection{Codigo}
\lstinputlisting[language=C++, breaklines=true]{../src/greedy/greedy.cpp}
\newpage

\section{Heurística de Búsqueda Local}

\subsection{Algoritmo}

El algoritmo propuesto sigue los siguientes pasos:

\begin{enumerate}
\item Utilizando la heuristica constructiva golosa, armamos una potencial solución factible $s \in S$. También es posible tomar la primera solución encontrada por backtracking.
\item A continuacion, construimos un posible vecino $s' \in N(s)$. Para ello, primero tomamos un nodo del cubrimiento, y lo sacamos.
\item Tomamos todos los nodos adyacentes a ese nodo, y los agregamos al conjunto solución.
\item Para mantener la independencia, quitamos todos los nodos adyacentes a los nuevos nodos.
\item A medida que recorremos los nodos, vamos marcando si los hemos manipulado durante el procedimiento.
\item Si la nueva solución $s'$ es mejor, intercambiamos la solucion anterior $s$ por $s'$ y repetimos el paso 2. Caso contrario, finaliza el algoritmo.
\end{enumerate}

Notar que por cada componente conexa tenemos un solo vecino. La idea es ir probando con todas las componentes conexas. Si modificamos una vez un nodo, después cuando hagamos el proceso de tomar un nodo, lo tomamos de la lista de elementos que NO modificamos. Esto se ve claro en los ejemplos de la constructiva greedy.

\subsection{Complejidad}
\newpage
%\subsection{Experimentacion}

Para la experimentacion se siguio con la metodologia indicada anteriormente. Los resultados fueron los siguientes.

\subsubsection{Heuristica Constructiva Golosa por Scoring con Busqueda Local por primer Vecinidad}

Los resultados temporales obtenidos fueron los siguientes:

\begin{figure}[H]
\centering

\begin{subfigure}[b]{0.4\textwidth}
	\includegraphics[scale=0.6]{graph/{output_local_1_1_n.csvTime}.pdf}
	\begin{center}
	Grafos Aleatorios ($m = n$)
	\end{center}
\end{subfigure}
\begin{subfigure}[b]{0.4\textwidth}
	\includegraphics[scale=0.6]{graph/{output_local_1_1_2n.csvTime}.pdf}
	\begin{center}
	Grafos Aleatorios ($m = 2n$)
	\end{center}
\end{subfigure}

\begin{subfigure}[b]{0.4\textwidth}
	\includegraphics[scale=0.6]{graph/{output_local_1_1_n2.csvTime}.pdf}
	\begin{center}
	Grafos Aleatorios ($m = \frac{n}{2}$)
	\end{center}
\end{subfigure}
\begin{subfigure}[b]{0.4\textwidth}
	\includegraphics[scale=0.6]{graph/{output_local_1_3_n4.csvTime}.pdf}
	\begin{center}
	Grafos Bipartitos ($\frac{n}{4}$ nodos en la segunda componente)
	\end{center}
\end{subfigure}
\end{figure}

\begin{figure}[H]
\centering

\begin{subfigure}[b]{0.4\textwidth}
	\includegraphics[scale=0.6]{graph/{output_local_1_3_3n4.csvTime}.pdf}
	\begin{center}
	Grafos Bipartitos ($\frac{3n}{4}$ nodos en la segunda componente)
	\end{center}
\end{subfigure}
\begin{subfigure}[b]{0.4\textwidth}
	\includegraphics[scale=0.6]{graph/{output_local_1_2_n4.csvTime}.pdf}
	\begin{center}
	Grafos $d$-regulares ($d = \frac{n}{4}$)
	\end{center}
\end{subfigure}

\begin{subfigure}[b]{0.4\textwidth}
	\includegraphics[scale=0.6]{graph/{output_local_1_2_n2.csvTime}.pdf}
	\begin{center}
	Grafos $d$-regulares ($m = \frac{n}{2}$)
	\end{center}
\end{subfigure}
\begin{subfigure}[b]{0.4\textwidth}
	\includegraphics[scale=0.6]{graph/{output_local_1_2_3n4.csvTime}.pdf}
	\begin{center}
	Grafos $d$-regulares ($m = \frac{3n}{4}$)
	\end{center}
\end{subfigure}
\end{figure}

\newpage
\begin{figure}[H]
\centering

\begin{subfigure}[b]{0.4\textwidth}
	\includegraphics[scale=0.6]{graph/{output_local_1_4_arbol.csvTime}.pdf}
	\begin{center}
	Arboles Binarios
	\end{center}
\end{subfigure}
\begin{subfigure}[b]{0.4\textwidth}
	\includegraphics[scale=0.6]{graph/{output_local_1_1_clique.csvTime}.pdf}
	\begin{center}
	Clique
	\end{center}
\end{subfigure}
\begin{center}
Greedy Heap
\end{center}
\end{figure}

\newpage

Para el analisis del tamaño de la solucion, vamos a ver los resultados por cada familia. En el caso de los aleatorios, los resultados para estas configuraciones fueron los siguiente:

\begin{table}[H]
\centering
\caption{Grafos aleatorios}
\label{my-label}
\begin{tabular}{|l|lll|}
\hline
        & \multicolumn{1}{l|}{m = n/2} & \multicolumn{1}{l|}{m = n} & m = 2n \\ \hline
n = 40  & 26                           & 21                         & 12     \\ \cline{1-1}
n = 60  & 38                           & 27                         & 16     \\ \cline{1-1}
n = 80  & 49                           & 33                         & 21     \\ \cline{1-1}
n = 100 & 59                           & 42                         & 24     \\ \cline{1-1}
n = 120 & 74                           & 55                         & 28     \\ \hline
\end{tabular}
\end{table}

Los tamaños obtenidos en el caso aleatorio son los mismo que los obtenidos mediante la solucion inicial, es decir, la busqueda local no mejoro ninguna de las soluciones.

Para los Grafos Bipartitos, los $d$-regulares y las cliques, el algoritmo encontro la solucion optima en todos los casos. Respecto a los arboles, la solucion del algoritmo siempre respeto la cota y el resultado fue el menor posible.

\newpage
\subsubsection{Heuristica Constructiva Golosa por Scoring con Busqueda Local por segunda Vecinidad}

Los resultados temporales obtenidos fueron los siguientes:

\begin{figure}[H]
\centering

\begin{subfigure}[b]{0.4\textwidth}
	\includegraphics[scale=0.6]{graph/{output_local_2_1_n.csvTime}.pdf}
	\begin{center}
	Grafos Aleatorios ($m = n$)
	\end{center}
\end{subfigure}
\begin{subfigure}[b]{0.4\textwidth}
	\includegraphics[scale=0.6]{graph/{output_local_2_1_2n.csvTime}.pdf}
	\begin{center}
	Grafos Aleatorios ($m = 2n$)
	\end{center}
\end{subfigure}

\begin{subfigure}[b]{0.4\textwidth}
	\includegraphics[scale=0.6]{graph/{output_local_2_1_n2.csvTime}.pdf}
	\begin{center}
	Grafos Aleatorios ($m = \frac{n}{2}$)
	\end{center}
\end{subfigure}
\begin{subfigure}[b]{0.4\textwidth}
	\includegraphics[scale=0.6]{graph/{output_local_2_3_n4.csvTime}.pdf}
	\begin{center}
	Grafos Bipartitos ($\frac{n}{4}$ nodos en la segunda componente)
	\end{center}
\end{subfigure}
\end{figure}

\begin{figure}[H]
\centering

\begin{subfigure}[b]{0.4\textwidth}
	\includegraphics[scale=0.6]{graph/{output_local_2_3_3n4.csvTime}.pdf}
	\begin{center}
	Grafos Bipartitos ($\frac{3n}{4}$ nodos en la segunda componente)
	\end{center}
\end{subfigure}
\begin{subfigure}[b]{0.4\textwidth}
	\includegraphics[scale=0.6]{graph/{output_local_2_2_n4.csvTime}.pdf}
	\begin{center}
	Grafos $d$-regulares ($d = \frac{n}{4}$)
	\end{center}
\end{subfigure}

\begin{subfigure}[b]{0.4\textwidth}
	\includegraphics[scale=0.6]{graph/{output_local_2_2_n2.csvTime}.pdf}
	\begin{center}
	Grafos $d$-regulares ($m = \frac{n}{2}$)
	\end{center}
\end{subfigure}
\begin{subfigure}[b]{0.4\textwidth}
	\includegraphics[scale=0.6]{graph/{output_local_2_2_3n4.csvTime}.pdf}
	\begin{center}
	Grafos $d$-regulares ($m = \frac{3n}{4}$)
	\end{center}
\end{subfigure}
\end{figure}

\newpage
\begin{figure}[H]
\centering

\begin{subfigure}[b]{0.4\textwidth}
	\includegraphics[scale=0.6]{graph/{output_local_2_4_arbol.csvTime}.pdf}
	\begin{center}
	Arboles Binarios
	\end{center}
\end{subfigure}
\begin{subfigure}[b]{0.4\textwidth}
	\includegraphics[scale=0.6]{graph/{output_local_2_1_clique.csvTime}.pdf}
	\begin{center}
	Clique
	\end{center}
\end{subfigure}
\begin{center}
Greedy Heap
\end{center}
\end{figure}

\newpage
Para el analisis del tamaño de la solucion, vamos a ver los resultados por cada familia. En el caso de los aleatorios, los resultados para estas configuraciones fueron los siguiente:

\begin{table}[H]
\centering
\caption{Grafos aleatorios}
\label{my-label}
\begin{tabular}{|l|lll|}
\hline
        & \multicolumn{1}{l|}{m = n/2} & \multicolumn{1}{l|}{m = n} & m = 2n \\ \hline
n = 40  & 26                           & 21                         & 12     \\ \cline{1-1}
n = 60  & 38                           & 27                         & 16     \\ \cline{1-1}
n = 80  & 49                           & 33                         & 21     \\ \cline{1-1}
n = 100 & 59                           & 42                         & 24     \\ \cline{1-1}
n = 120 & 74                           & 55                         & 28     \\ \hline
\end{tabular}
\end{table}

Al igual que con la primer vecinidad, no se pudo mejorar la solucion original. Respecto al resto de las familias, las soluciones fueron optimas y los resultados fueron cercanos a las cotas de cada familia.

\newpage
\subsubsection{Heuristica Constructiva Golosa por Grado con Busqueda Local por primer Vecinidad}

Los resultados temporales obtenidos fueron los siguientes:

\begin{figure}[H]
\centering

\begin{subfigure}[b]{0.4\textwidth}
	\includegraphics[scale=0.6]{graph/{output_local_3_1_n.csvTime}.pdf}
	\begin{center}
	Grafos Aleatorios ($m = n$)
	\end{center}
\end{subfigure}
\begin{subfigure}[b]{0.4\textwidth}
	\includegraphics[scale=0.6]{graph/{output_local_3_1_2n.csvTime}.pdf}
	\begin{center}
	Grafos Aleatorios ($m = 2n$)
	\end{center}
\end{subfigure}

\begin{subfigure}[b]{0.4\textwidth}
	\includegraphics[scale=0.6]{graph/{output_local_3_1_n2.csvTime}.pdf}
	\begin{center}
	Grafos Aleatorios ($m = \frac{n}{2}$)
	\end{center}
\end{subfigure}
\begin{subfigure}[b]{0.4\textwidth}
	\includegraphics[scale=0.6]{graph/{output_local_3_3_n4.csvTime}.pdf}
	\begin{center}
	Grafos Bipartitos ($\frac{n}{4}$ nodos en la segunda componente)
	\end{center}
\end{subfigure}
\end{figure}

\begin{figure}[H]
\centering

\begin{subfigure}[b]{0.4\textwidth}
	\includegraphics[scale=0.6]{graph/{output_local_3_3_3n4.csvTime}.pdf}
	\begin{center}
	Grafos Bipartitos ($\frac{3n}{4}$ nodos en la segunda componente)
	\end{center}
\end{subfigure}
\begin{subfigure}[b]{0.4\textwidth}
	\includegraphics[scale=0.6]{graph/{output_local_3_2_n4.csvTime}.pdf}
	\begin{center}
	Grafos $d$-regulares ($d = \frac{n}{4}$)
	\end{center}
\end{subfigure}

\begin{subfigure}[b]{0.4\textwidth}
	\includegraphics[scale=0.6]{graph/{output_local_3_2_n2.csvTime}.pdf}
	\begin{center}
	Grafos $d$-regulares ($m = \frac{n}{2}$)
	\end{center}
\end{subfigure}
\begin{subfigure}[b]{0.4\textwidth}
	\includegraphics[scale=0.6]{graph/{output_local_3_2_3n4.csvTime}.pdf}
	\begin{center}
	Grafos $d$-regulares ($m = \frac{3n}{4}$)
	\end{center}
\end{subfigure}
\end{figure}

\newpage
\begin{figure}[H]
\centering

\begin{subfigure}[b]{0.4\textwidth}
	\includegraphics[scale=0.6]{graph/{output_local_3_4_arbol.csvTime}.pdf}
	\begin{center}
	Arboles Binarios
	\end{center}
\end{subfigure}
\begin{subfigure}[b]{0.4\textwidth}
	\includegraphics[scale=0.6]{graph/{output_local_3_1_clique.csvTime}.pdf}
	\begin{center}
	Clique
	\end{center}
\end{subfigure}
\begin{center}
Greedy Heap
\end{center}
\end{figure}

\newpage
Para el analisis del tamaño de la solucion, vamos a ver los resultados por cada familia. En el caso de los aleatorios, los resultados para estas configuraciones fueron los siguiente:

\begin{table}[H]
\centering
\caption{Grafos Aleatorios}
\label{my-label}
\begin{tabular}{|l|lll|}
\hline
        & \multicolumn{1}{l|}{m = n/2} & \multicolumn{1}{l|}{m = n} & m = 2n \\ \hline
n = 40  & 32                           & 26                         & 16     \\ \cline{1-1}
n = 60  & 43                           & 33                         & 22     \\ \cline{1-1}
n = 80  & 56                           & 44                         & 30     \\ \cline{1-1}
n = 100 & 67                           & 56                         & 40     \\ \cline{1-1}
n = 120 & 74                           & 66                         & 46     \\ \hline
\end{tabular}
\end{table}

Al igual que con el primer criterio de vecinidad, en el caso de los aleatorios la solucion no mejoro, se mantuvo en los mismo valores de la original. Sin embargo, este presento mejoras en los Arboles, dando una solucion menor.

\newpage
\subsubsection{Heuristica Constructiva Golosa por Grado con Busqueda Local por segunda Vecinidad}

Los resultados temporales obtenidos fueron los siguientes:

\begin{figure}[H]
\centering

\begin{subfigure}[b]{0.4\textwidth}
	\includegraphics[scale=0.6]{graph/{output_local_4_1_n.csvTime}.pdf}
	\begin{center}
	Grafos Aleatorios ($m = n$)
	\end{center}
\end{subfigure}
\begin{subfigure}[b]{0.4\textwidth}
	\includegraphics[scale=0.6]{graph/{output_local_4_1_2n.csvTime}.pdf}
	\begin{center}
	Grafos Aleatorios ($m = 2n$)
	\end{center}
\end{subfigure}

\begin{subfigure}[b]{0.4\textwidth}
	\includegraphics[scale=0.6]{graph/{output_local_4_1_n2.csvTime}.pdf}
	\begin{center}
	Grafos Aleatorios ($m = \frac{n}{2}$)
	\end{center}
\end{subfigure}
\begin{subfigure}[b]{0.4\textwidth}
	\includegraphics[scale=0.6]{graph/{output_local_4_3_n4.csvTime}.pdf}
	\begin{center}
	Grafos Bipartitos ($\frac{n}{4}$ nodos en la segunda componente)
	\end{center}
\end{subfigure}
\end{figure}

\begin{figure}[H]
\centering

\begin{subfigure}[b]{0.4\textwidth}
	\includegraphics[scale=0.6]{graph/{output_local_4_3_3n4.csvTime}.pdf}
	\begin{center}
	Grafos Bipartitos ($\frac{3n}{4}$ nodos en la segunda componente)
	\end{center}
\end{subfigure}
\begin{subfigure}[b]{0.4\textwidth}
	\includegraphics[scale=0.6]{graph/{output_local_4_2_n4.csvTime}.pdf}
	\begin{center}
	Grafos $d$-regulares ($d = \frac{n}{4}$)
	\end{center}
\end{subfigure}

\begin{subfigure}[b]{0.4\textwidth}
	\includegraphics[scale=0.6]{graph/{output_local_4_2_n2.csvTime}.pdf}
	\begin{center}
	Grafos $d$-regulares ($m = \frac{n}{2}$)
	\end{center}
\end{subfigure}
\begin{subfigure}[b]{0.4\textwidth}
	\includegraphics[scale=0.6]{graph/{output_local_4_2_3n4.csvTime}.pdf}
	\begin{center}
	Grafos $d$-regulares ($m = \frac{3n}{4}$)
	\end{center}
\end{subfigure}
\end{figure}

\newpage
\begin{figure}[H]
\centering

\begin{subfigure}[b]{0.4\textwidth}
	\includegraphics[scale=0.6]{graph/{output_local_4_4_arbol.csvTime}.pdf}
	\begin{center}
	Arboles Binarios
	\end{center}
\end{subfigure}
\begin{subfigure}[b]{0.4\textwidth}
	\includegraphics[scale=0.6]{graph/{output_local_4_1_clique.csvTime}.pdf}
	\begin{center}
	Clique
	\end{center}
\end{subfigure}
\begin{center}
Greedy Heap
\end{center}
\end{figure}

\newpage
Para el analisis del tamaño de la solucion, vamos a ver los resultados por cada familia. En el caso de los aleatorios, los resultados para estas configuraciones fueron los siguiente:

\begin{table}[H]
\centering
\caption{Grafos Aleatorios}
\label{my-label}
\begin{tabular}{|l|lll|}
\hline
        & \multicolumn{1}{l|}{m = n/2} & \multicolumn{1}{l|}{m = n} & m = 2n \\ \hline
n = 40  & 27                           & 20                         & 16     \\ \cline{1-1}
n = 60  & 40                           & 33                         & 22     \\ \cline{1-1}
n = 80  & 52                           & 42                         & 27     \\ \cline{1-1}
n = 100 & 65                           & 51                         & 31     \\ \cline{1-1}
n = 120 & 80                           & 62                         & 40     \\ \hline
\end{tabular}
\end{table}

A diferencia de los otros casos, aqui podemos ver como la segunda vecinidad mejoro amplimente el resultado anterior, a tal punto que se llego a un mejor resultado que el obtenido con la Heuristica Constructiva Golosa por Scoring.

\subsubsection{Conclusion}

Primero vamos a ver los resultados por cada familia.

\begin{itemize}
	\item Grafos Aleatorios: Como era de esperar la cantidad de conexiones volvio a impactar en el tiempo de cada algoritmo. Tambien pudimos apreciar el costo agregado del segundo criterio de vecinidad, ya que en los casos donde se lo aplico, los tiempos aumentaron de manera considerable.
	\item Grafos Bipartitos: El tiempo de convergencia al aplicar el segundo criterio de vecinidad aumento considerablemente, esto es un detrimento importante, ya que la solucion inicial no fue mejorada.
	\item Grafos $d$-regulares: Aqui tambien los tiempos de ejecucion aumentaron, sin embargo, el aumento no fue tan pronunciado como en las dos familias anteriores.
	\item Arboles binarios: A diferencia de los casos anteriores, los tiempos obtenidos aqui no difirien mucho entre vecinidades. Estas tampoco pudieron lograr una mejora importante respecto a la solucion original.
	\item Cliques: En las cliques sabemos que toda solucion puede tener a lo sumo un nodo, con lo cual la misma no puede ser mejorada. Los dos criterios de vecinidad tomaron tiempos similares de ejecucion.
\end{itemize}

La eficiencia del primer criterio de vecinidad fue limitada, si bien el mismo no introducia mucho tiempo extra en la ejecucion del algoritmo, no podia mejorar las solucion por un margen razonable. Por el otro lado, el segundo criterio de vecinidad logro demostrar su efectividad para poder mejorar soluciones, pero esto tuvo un costo temporal grande. Consideramos que la mejor heuristicas de las presentadas en este punto es la Heuristica Constructiva Golosa por Grado con Busqueda Local por segunda Vecinidad, si bien la Busqueda Local agrega una cantidad considerable de tiempo a la ejecucion, los resultados mejorar ampliamente, superando includo los de la Heuristica Constructiva Golosa por Scoring.
\newpage

\section{Metaheurística GRASP}

\subsection{Algoritmo}

GRASP (Greedy Randomized Adaptative Search Procedure) es una combinación entre una heurística golosa aleatorizada y un procedimiento de búsqueda local. La metaheurística se puede ver con el siguiente pseudocódigo:

\begin{algorithmic}
\Procedure{grasp}{G, k}
\State{G bestSolution}
\While{$!terminationCondition(G, k, bestSolution)$}
	\State{s $\gets$ randomGreedyHeuristic(G, k)}
	\State{s $\gets$ localSearch(G,s)}
	\If{$|s| < |bestSolution|$}
	\State{bestSolution $\gets$ s}
	\EndIf
\EndWhile
\EndProcedure
\end{algorithmic}

De este procedimiento surgen dos preguntas, que en realidad son cosas que debemos definir. De donde proviene la aleatoriedad de la heurística greedy? Cual es criterio de terminación que utilizaremos?

\subsection{Random Greedy Heuristic}

\subsubsection{Por cantidad}

Para agregarle una componente aleatoria a \texttt{GRASP}, se propone fabricar en cada paso de la heurística constructiva golosa una \textit{Lista Restricta de Candidatos} (RCL) y elegir aleatoriamente un candidato de esta lista. Para ello, decidimos crear la función \texttt{greedyHeapConstructiveRandomized(Node graph[], int n, int k)} que lo que hace es ir eligiendo aleatoriamente de  los $k$ vértices con mayor grado utilizando un heap como estructura auxiliar. Esto se puede ver en el siguiente pseudocódigo:

\begin{algorithmic}
\Procedure{greedyHeapConstructiveRandomized}{G,k}

\State{nodeHeap $\gets$ buildHeap(G.V)}

\While{!nodeHeap.isEmpty()}
	\State{node $\gets$ nodeHeap.topRandomPop(k)}
	\If{node.reachable == true}
		\State{continue}
	\EndIf
	\State{node.added = true}
	
	\ForAll{$adj \in node.adj$}
		\State{adj.reachable $\gets$ true}
	\EndFor
\EndWhile
\EndProcedure
\end{algorithmic}

\subsubsection{Por valor}

Al igual que en el criterio anterior, elegimos un candidato aleatorio de una lista desde un heap. Sin embargo, ahora un vértice está en la lista de candidatos si y sólo si el grado de cualquier nodo en la lista esta a una distancia menor o igual a $k$ grados del vértice de mayor grado disponible en el heap. Esto se puede ver en el siguiente pseudocódigo:

\begin{algorithmic}
\Procedure{greedyHeapConstructiveRandomized}{G,k}

\State{nodeHeap $\gets$ buildHeap(G.V)}

\While{!nodeHeap.isEmpty()}
	\State{node $\gets$ nodeHeap.topRandomPopByValue(k)}
	\If{node.reachable == true}
		\State{continue}
	\EndIf
	\State{node.added = true}
	
	\ForAll{$adj \in node.adj$}
		\State{adj.reachable $\gets$ true}
	\EndFor
\EndWhile
\EndProcedure
\end{algorithmic}

\subsection{Criterios de terminación}
\begin{enumerate}
\item No se encontró ninguna mejora en las ultimas $j$ iteraciones.
\item Se alcanzo un límite prefijado de  $j$ iteraciones.
\end{enumerate}
\newpage
%\subsection{Experimentacion}

Debido a la longitud de los nombres, se utilizo siguiente lista para referirnos a cada una de las configuraciones:

\begin{itemize}
	\item GRASP1: Greedy por cantidad, Busqueda Local con primer criterio de vecinidad, Terminacion sin mejoras
	\item GRASP2: Greedy por valor, Busqueda Local con primer criterio de vecinidad, Terminacion sin mejoras
	\item GRASP3: Greedy por cantidad, Busqueda Local con segundo criterio de vecinidad, Terminacion sin mejoras
	\item GRASP4: Greedy por cantidad, Busqueda Local con segundo criterio de vecinidad, Terminacion sin mejoras
	\item GRASP5: Greedy por valor, Busqueda Local con primer criterio de vecinidad, Terminacion prefijada
	\item GRASP6: Greedy por cantidad, Busqueda Local con primer criterio de vecinidad, Terminacion prefijada
	\item GRASP7: Greedy por cantidad, Busqueda Local con segundo criterio de vecinidad, Terminacion prefijada
	\item GRASP8: Greedy por valor, Busqueda Local con segundo criterio de vecinidad, Terminacion prefijada
\end{itemize}

Para la experimentacion se siguio con la metodologia indicada anteriormente. Una vez determinada la mejor configuracion de GRASP, se procedio a la calibracion de parametros de la misma.

\newpage
\subsubsection{GRASP1}

Los resultados temporales obtenidos fueron los siguientes:

\begin{figure}[H]
\centering

\begin{subfigure}[b]{0.4\textwidth}
	\includegraphics[scale=0.6]{graph/{output_grasp_1_1_n.csvTime}.pdf}
	\begin{center}
	Grafos Aleatorios ($m = n$)
	\end{center}
\end{subfigure}
\begin{subfigure}[b]{0.4\textwidth}
	\includegraphics[scale=0.6]{graph/{output_grasp_1_1_2n.csvTime}.pdf}
	\begin{center}
	Grafos Aleatorios ($m = 2n$)
	\end{center}
\end{subfigure}

\begin{subfigure}[b]{0.4\textwidth}
	\includegraphics[scale=0.6]{graph/{output_grasp_1_1_n2.csvTime}.pdf}
	\begin{center}
	Grafos Aleatorios ($m = \frac{n}{2}$)
	\end{center}
\end{subfigure}
\begin{subfigure}[b]{0.4\textwidth}
	\includegraphics[scale=0.6]{graph/{output_grasp_1_3_n4.csvTime}.pdf}
	\begin{center}
	Grafos Bipartitos ($\frac{n}{4}$ nodos en la segunda componente)
	\end{center}
\end{subfigure}
\end{figure}

\begin{figure}[H]
\centering

\begin{subfigure}[b]{0.4\textwidth}
	\includegraphics[scale=0.6]{graph/{output_grasp_1_3_3n4.csvTime}.pdf}
	\begin{center}
	Grafos Bipartitos ($\frac{3n}{4}$ nodos en la segunda componente)
	\end{center}
\end{subfigure}
\begin{subfigure}[b]{0.4\textwidth}
	\includegraphics[scale=0.6]{graph/{output_grasp_1_2_n4.csvTime}.pdf}
	\begin{center}
	Grafos $d$-regulares ($d = \frac{n}{4}$)
	\end{center}
\end{subfigure}

\begin{subfigure}[b]{0.4\textwidth}
	\includegraphics[scale=0.6]{graph/{output_grasp_1_2_n2.csvTime}.pdf}
	\begin{center}
	Grafos $d$-regulares ($m = \frac{n}{2}$)
	\end{center}
\end{subfigure}
\begin{subfigure}[b]{0.4\textwidth}
	\includegraphics[scale=0.6]{graph/{output_grasp_1_2_3n4.csvTime}.pdf}
	\begin{center}
	Grafos $d$-regulares ($m = \frac{3n}{4}$)
	\end{center}
\end{subfigure}
\end{figure}

\newpage
\begin{figure}[H]
\centering

\begin{subfigure}[b]{0.4\textwidth}
	\includegraphics[scale=0.6]{graph/{output_grasp_1_4_arbol.csvTime}.pdf}
	\begin{center}
	Arboles Binarios
	\end{center}
\end{subfigure}
\begin{subfigure}[b]{0.4\textwidth}
	\includegraphics[scale=0.6]{graph/{output_grasp_1_1_clique.csvTime}.pdf}
	\begin{center}
	Clique
	\end{center}
\end{subfigure}
\end{figure}

Para el analisis del tamaño de la solucion, vamos a ver los resultados por cada familia. En el caso de los aleatorios, los resultados para estas configuraciones fueron los siguiente:

\begin{table}[H]
\centering
\caption{Grafos aleatorios}
\label{my-label}
\begin{tabular}{|l|lll|}
\hline
        & \multicolumn{1}{l|}{m = n/2} & \multicolumn{1}{l|}{m = n} & m = 2n \\ \hline
n = 40  & 26                           & 21                         & 15     \\ \cline{1-1}
n = 60  & 39                           & 32                         & 21     \\ \cline{1-1}
n = 80  & 49                           & 34                         & 25     \\ \cline{1-1}
n = 100 & 60                           & 49                         & 32     \\ \cline{1-1}
n = 120 & 75                           & 55                         & 39     \\ \hline
\end{tabular}
\end{table}

Para los arboles $d$-regulares y Arboles Binarios, la heuristica GRASP comenzo a mostrar resultados poco eficientes, elgiendo una mayor cantidad de nodos que la habitual. Esta tendencia se da para todas las configuraciones de GRASP.

\newpage
\subsubsection{GRASP2}

Los resultados temporales obtenidos fueron los siguientes:

\begin{figure}[H]
\centering

\begin{subfigure}[b]{0.4\textwidth}
	\includegraphics[scale=0.6]{graph/{output_grasp_2_1_n.csvTime}.pdf}
	\begin{center}
	Grafos Aleatorios ($m = n$)
	\end{center}
\end{subfigure}
\begin{subfigure}[b]{0.4\textwidth}
	\includegraphics[scale=0.6]{graph/{output_grasp_2_1_2n.csvTime}.pdf}
	\begin{center}
	Grafos Aleatorios ($m = 2n$)
	\end{center}
\end{subfigure}

\begin{subfigure}[b]{0.4\textwidth}
	\includegraphics[scale=0.6]{graph/{output_grasp_2_1_n2.csvTime}.pdf}
	\begin{center}
	Grafos Aleatorios ($m = \frac{n}{2}$)
	\end{center}
\end{subfigure}
\begin{subfigure}[b]{0.4\textwidth}
	\includegraphics[scale=0.6]{graph/{output_grasp_2_3_n4.csvTime}.pdf}
	\begin{center}
	Grafos Bipartitos ($\frac{n}{4}$ nodos en la segunda componente)
	\end{center}
\end{subfigure}
\end{figure}

\begin{figure}[H]
\centering

\begin{subfigure}[b]{0.4\textwidth}
	\includegraphics[scale=0.6]{graph/{output_grasp_2_3_3n4.csvTime}.pdf}
	\begin{center}
	Grafos Bipartitos ($\frac{3n}{4}$ nodos en la segunda componente)
	\end{center}
\end{subfigure}
\begin{subfigure}[b]{0.4\textwidth}
	\includegraphics[scale=0.6]{graph/{output_grasp_2_2_n4.csvTime}.pdf}
	\begin{center}
	Grafos $d$-regulares ($d = \frac{n}{4}$)
	\end{center}
\end{subfigure}

\begin{subfigure}[b]{0.4\textwidth}
	\includegraphics[scale=0.6]{graph/{output_grasp_2_2_n2.csvTime}.pdf}
	\begin{center}
	Grafos $d$-regulares ($m = \frac{n}{2}$)
	\end{center}
\end{subfigure}
\begin{subfigure}[b]{0.4\textwidth}
	\includegraphics[scale=0.6]{graph/{output_grasp_2_2_3n4.csvTime}.pdf}
	\begin{center}
	Grafos $d$-regulares ($m = \frac{3n}{4}$)
	\end{center}
\end{subfigure}
\end{figure}

\newpage
\begin{figure}[H]
\centering

\begin{subfigure}[b]{0.4\textwidth}
	\includegraphics[scale=0.6]{graph/{output_grasp_2_4_arbol.csvTime}.pdf}
	\begin{center}
	Arboles Binarios
	\end{center}
\end{subfigure}
\begin{subfigure}[b]{0.4\textwidth}
	\includegraphics[scale=0.6]{graph/{output_grasp_2_1_clique.csvTime}.pdf}
	\begin{center}
	Clique
	\end{center}
\end{subfigure}
\end{figure}

Para el analisis del tamaño de la solucion, vamos a ver los resultados por cada familia. En el caso de los aleatorios, los resultados para estas configuraciones fueron los siguiente:

\begin{table}[H]
\centering
\caption{Grafos aleatorios}
\label{my-label}
\begin{tabular}{|l|lll|}
\hline
        & \multicolumn{1}{l|}{m = n/2} & \multicolumn{1}{l|}{m = n} & m = 2n \\ \hline
n = 40  & 29                           & 23                         & 15     \\ \cline{1-1}
n = 60  & 40                           & 34                         & 22     \\ \cline{1-1}
n = 80  & 53                           & 39                         & 29     \\ \cline{1-1}
n = 100 & 62                           & 51                         & 34     \\ \cline{1-1}
n = 120 & 80                           & 59                         & 41     \\ \hline
\end{tabular}
\end{table}

\newpage
\subsubsection{GRASP3}

Los resultados temporales obtenidos fueron los siguientes:

\begin{figure}[H]
\centering

\begin{subfigure}[b]{0.4\textwidth}
	\includegraphics[scale=0.6]{graph/{output_grasp_3_1_n.csvTime}.pdf}
	\begin{center}
	Grafos Aleatorios ($m = n$)
	\end{center}
\end{subfigure}
\begin{subfigure}[b]{0.4\textwidth}
	\includegraphics[scale=0.6]{graph/{output_grasp_3_1_2n.csvTime}.pdf}
	\begin{center}
	Grafos Aleatorios ($m = 2n$)
	\end{center}
\end{subfigure}

\begin{subfigure}[b]{0.4\textwidth}
	\includegraphics[scale=0.6]{graph/{output_grasp_3_1_n2.csvTime}.pdf}
	\begin{center}
	Grafos Aleatorios ($m = \frac{n}{2}$)
	\end{center}
\end{subfigure}
\begin{subfigure}[b]{0.4\textwidth}
	\includegraphics[scale=0.6]{graph/{output_grasp_3_3_n4.csvTime}.pdf}
	\begin{center}
	Grafos Bipartitos ($\frac{n}{4}$ nodos en la segunda componente)
	\end{center}
\end{subfigure}
\end{figure}

\begin{figure}[H]
\centering

\begin{subfigure}[b]{0.4\textwidth}
	\includegraphics[scale=0.6]{graph/{output_grasp_3_3_3n4.csvTime}.pdf}
	\begin{center}
	Grafos Bipartitos ($\frac{3n}{4}$ nodos en la segunda componente)
	\end{center}
\end{subfigure}
\begin{subfigure}[b]{0.4\textwidth}
	\includegraphics[scale=0.6]{graph/{output_grasp_3_2_n4.csvTime}.pdf}
	\begin{center}
	Grafos $d$-regulares ($d = \frac{n}{4}$)
	\end{center}
\end{subfigure}

\begin{subfigure}[b]{0.4\textwidth}
	\includegraphics[scale=0.6]{graph/{output_grasp_3_2_n2.csvTime}.pdf}
	\begin{center}
	Grafos $d$-regulares ($m = \frac{n}{2}$)
	\end{center}
\end{subfigure}
\begin{subfigure}[b]{0.4\textwidth}
	\includegraphics[scale=0.6]{graph/{output_grasp_3_2_3n4.csvTime}.pdf}
	\begin{center}
	Grafos $d$-regulares ($m = \frac{3n}{4}$)
	\end{center}
\end{subfigure}
\end{figure}

\newpage
\begin{figure}[H]
\centering

\begin{subfigure}[b]{0.4\textwidth}
	\includegraphics[scale=0.6]{graph/{output_grasp_3_4_arbol.csvTime}.pdf}
	\begin{center}
	Arboles Binarios
	\end{center}
\end{subfigure}
\begin{subfigure}[b]{0.4\textwidth}
	\includegraphics[scale=0.6]{graph/{output_grasp_3_1_clique.csvTime}.pdf}
	\begin{center}
	Clique
	\end{center}
\end{subfigure}
\end{figure}

Para el analisis del tamaño de la solucion, vamos a ver los resultados por cada familia. En el caso de los aleatorios, los resultados para estas configuraciones fueron los siguiente:

\begin{table}[H]
\centering
\caption{Grafos Aleatorios}
\label{my-label}
\begin{tabular}{|l|lll|}
\hline
        & \multicolumn{1}{l|}{m = n/2} & \multicolumn{1}{l|}{m = n} & m = 2n \\ \hline
n = 40  & 26                           & 21                         & 14     \\ \cline{1-1}
n = 60  & 39                           & 32                         & 19     \\ \cline{1-1}
n = 80  & 49                           & 34                         & 24     \\ \cline{1-1}
n = 100 & 60                           & 47                         & 30     \\ \cline{1-1}
n = 120 & 75                           & 55                         & 37     \\ \hline
\end{tabular}
\end{table}

\newpage
\subsubsection{GRASP4}

Los resultados temporales obtenidos fueron los siguientes:

\begin{figure}[H]
\centering

\begin{subfigure}[b]{0.4\textwidth}
	\includegraphics[scale=0.6]{graph/{output_grasp_4_1_n.csvTime}.pdf}
	\begin{center}
	Grafos Aleatorios ($m = n$)
	\end{center}
\end{subfigure}
\begin{subfigure}[b]{0.4\textwidth}
	\includegraphics[scale=0.6]{graph/{output_grasp_4_1_2n.csvTime}.pdf}
	\begin{center}
	Grafos Aleatorios ($m = 2n$)
	\end{center}
\end{subfigure}

\begin{subfigure}[b]{0.4\textwidth}
	\includegraphics[scale=0.6]{graph/{output_grasp_4_1_n2.csvTime}.pdf}
	\begin{center}
	Grafos Aleatorios ($m = \frac{n}{2}$)
	\end{center}
\end{subfigure}
\begin{subfigure}[b]{0.4\textwidth}
	\includegraphics[scale=0.6]{graph/{output_grasp_4_3_n4.csvTime}.pdf}
	\begin{center}
	Grafos Bipartitos ($\frac{n}{4}$ nodos en la segunda componente)
	\end{center}
\end{subfigure}
\end{figure}

\begin{figure}[H]
\centering

\begin{subfigure}[b]{0.4\textwidth}
	\includegraphics[scale=0.6]{graph/{output_grasp_4_3_3n4.csvTime}.pdf}
	\begin{center}
	Grafos Bipartitos ($\frac{3n}{4}$ nodos en la segunda componente)
	\end{center}
\end{subfigure}
\begin{subfigure}[b]{0.4\textwidth}
	\includegraphics[scale=0.6]{graph/{output_grasp_4_2_n4.csvTime}.pdf}
	\begin{center}
	Grafos $d$-regulares ($d = \frac{n}{4}$)
	\end{center}
\end{subfigure}

\begin{subfigure}[b]{0.4\textwidth}
	\includegraphics[scale=0.6]{graph/{output_grasp_4_2_n2.csvTime}.pdf}
	\begin{center}
	Grafos $d$-regulares ($m = \frac{n}{2}$)
	\end{center}
\end{subfigure}
\begin{subfigure}[b]{0.4\textwidth}
	\includegraphics[scale=0.6]{graph/{output_grasp_4_2_3n4.csvTime}.pdf}
	\begin{center}
	Grafos $d$-regulares ($m = \frac{3n}{4}$)
	\end{center}
\end{subfigure}
\end{figure}

\newpage
\begin{figure}[H]
\centering

\begin{subfigure}[b]{0.4\textwidth}
	\includegraphics[scale=0.6]{graph/{output_grasp_4_4_arbol.csvTime}.pdf}
	\begin{center}
	Arboles Binarios
	\end{center}
\end{subfigure}
\begin{subfigure}[b]{0.4\textwidth}
	\includegraphics[scale=0.6]{graph/{output_grasp_4_1_clique.csvTime}.pdf}
	\begin{center}
	Clique
	\end{center}
\end{subfigure}
\end{figure}

Para el analisis del tamaño de la solucion, vamos a ver los resultados por cada familia. En el caso de los aleatorios, los resultados para estas configuraciones fueron los siguiente:

\begin{table}[H]
\centering
\caption{Grafos Aleatorios}
\label{my-label}
\begin{tabular}{|l|lll|}
\hline
        & \multicolumn{1}{l|}{m = n/2} & \multicolumn{1}{l|}{m = n} & m = 2n \\ \hline
n = 40  & 25                           & 19                         & 13     \\ \cline{1-1}
n = 60  & 40                           & 32                         & 18     \\ \cline{1-1}
n = 80  & 53                           & 39                         & 26     \\ \cline{1-1}
n = 100 & 60                           & 49                         & 31     \\ \cline{1-1}
n = 120 & 76                           & 55                         & 39     \\ \hline
\end{tabular}
\end{table}

\newpage
\subsubsection{GRASP5}

Los resultados temporales obtenidos fueron los siguientes:

\begin{figure}[H]
\centering

\begin{subfigure}[b]{0.4\textwidth}
	\includegraphics[scale=0.6]{graph/{output_grasp_5_1_n.csvTime}.pdf}
	\begin{center}
	Grafos Aleatorios ($m = n$)
	\end{center}
\end{subfigure}
\begin{subfigure}[b]{0.4\textwidth}
	\includegraphics[scale=0.6]{graph/{output_grasp_5_1_2n.csvTime}.pdf}
	\begin{center}
	Grafos Aleatorios ($m = 2n$)
	\end{center}
\end{subfigure}

\begin{subfigure}[b]{0.4\textwidth}
	\includegraphics[scale=0.6]{graph/{output_grasp_5_1_n2.csvTime}.pdf}
	\begin{center}
	Grafos Aleatorios ($m = \frac{n}{2}$)
	\end{center}
\end{subfigure}
\begin{subfigure}[b]{0.4\textwidth}
	\includegraphics[scale=0.6]{graph/{output_grasp_5_3_n4.csvTime}.pdf}
	\begin{center}
	Grafos Bipartitos ($\frac{n}{4}$ nodos en la segunda componente)
	\end{center}
\end{subfigure}
\end{figure}

\begin{figure}[H]
\centering

\begin{subfigure}[b]{0.4\textwidth}
	\includegraphics[scale=0.6]{graph/{output_grasp_5_3_3n4.csvTime}.pdf}
	\begin{center}
	Grafos Bipartitos ($\frac{3n}{4}$ nodos en la segunda componente)
	\end{center}
\end{subfigure}
\begin{subfigure}[b]{0.4\textwidth}
	\includegraphics[scale=0.6]{graph/{output_grasp_5_2_n4.csvTime}.pdf}
	\begin{center}
	Grafos $d$-regulares ($d = \frac{n}{4}$)
	\end{center}
\end{subfigure}

\begin{subfigure}[b]{0.4\textwidth}
	\includegraphics[scale=0.6]{graph/{output_grasp_5_2_n2.csvTime}.pdf}
	\begin{center}
	Grafos $d$-regulares ($m = \frac{n}{2}$)
	\end{center}
\end{subfigure}
\begin{subfigure}[b]{0.4\textwidth}
	\includegraphics[scale=0.6]{graph/{output_grasp_5_2_3n4.csvTime}.pdf}
	\begin{center}
	Grafos $d$-regulares ($m = \frac{3n}{4}$)
	\end{center}
\end{subfigure}
\end{figure}

\newpage
\begin{figure}[H]
\centering

\begin{subfigure}[b]{0.4\textwidth}
	\includegraphics[scale=0.6]{graph/{output_grasp_5_4_arbol.csvTime}.pdf}
	\begin{center}
	Arboles Binarios
	\end{center}
\end{subfigure}
\begin{subfigure}[b]{0.4\textwidth}
	\includegraphics[scale=0.6]{graph/{output_grasp_5_1_clique.csvTime}.pdf}
	\begin{center}
	Clique
	\end{center}
\end{subfigure}
\end{figure}

Para el analisis del tamaño de la solucion, vamos a ver los resultados por cada familia. En el caso de los aleatorios, los resultados para estas configuraciones fueron los siguiente:

\begin{table}[H]
\centering
\caption{Grafos aleatorios}
\label{my-label}
\begin{tabular}{|l|lll|}
\hline
        & \multicolumn{1}{l|}{m = n/2} & \multicolumn{1}{l|}{m = n} & m = 2n \\ \hline
n = 40  & 26                           & 22                         & 15     \\ \cline{1-1}
n = 60  & 40                           & 32                         & 20     \\ \cline{1-1}
n = 80  & 49                           & 35                         & 27     \\ \cline{1-1}
n = 100 & 60                           & 50                         & 35     \\ \cline{1-1}
n = 120 & 75                           & 56                         & 39     \\ \hline
\end{tabular}
\end{table}

\newpage
\subsubsection{GRASP6}

Los resultados temporales obtenidos fueron los siguientes:

\begin{figure}[H]
\centering

\begin{subfigure}[b]{0.4\textwidth}
	\includegraphics[scale=0.6]{graph/{output_grasp_6_1_n.csvTime}.pdf}
	\begin{center}
	Grafos Aleatorios ($m = n$)
	\end{center}
\end{subfigure}
\begin{subfigure}[b]{0.4\textwidth}
	\includegraphics[scale=0.6]{graph/{output_grasp_6_1_2n.csvTime}.pdf}
	\begin{center}
	Grafos Aleatorios ($m = 2n$)
	\end{center}
\end{subfigure}

\begin{subfigure}[b]{0.4\textwidth}
	\includegraphics[scale=0.6]{graph/{output_grasp_6_1_n2.csvTime}.pdf}
	\begin{center}
	Grafos Aleatorios ($m = \frac{n}{2}$)
	\end{center}
\end{subfigure}
\begin{subfigure}[b]{0.4\textwidth}
	\includegraphics[scale=0.6]{graph/{output_grasp_6_3_n4.csvTime}.pdf}
	\begin{center}
	Grafos Bipartitos ($\frac{n}{4}$ nodos en la segunda componente)
	\end{center}
\end{subfigure}
\end{figure}

\begin{figure}[H]
\centering

\begin{subfigure}[b]{0.4\textwidth}
	\includegraphics[scale=0.6]{graph/{output_grasp_6_3_3n4.csvTime}.pdf}
	\begin{center}
	Grafos Bipartitos ($\frac{3n}{4}$ nodos en la segunda componente)
	\end{center}
\end{subfigure}
\begin{subfigure}[b]{0.4\textwidth}
	\includegraphics[scale=0.6]{graph/{output_grasp_6_2_n4.csvTime}.pdf}
	\begin{center}
	Grafos $d$-regulares ($d = \frac{n}{4}$)
	\end{center}
\end{subfigure}

\begin{subfigure}[b]{0.4\textwidth}
	\includegraphics[scale=0.6]{graph/{output_grasp_6_2_n2.csvTime}.pdf}
	\begin{center}
	Grafos $d$-regulares ($m = \frac{n}{2}$)
	\end{center}
\end{subfigure}
\begin{subfigure}[b]{0.4\textwidth}
	\includegraphics[scale=0.6]{graph/{output_grasp_6_2_3n4.csvTime}.pdf}
	\begin{center}
	Grafos $d$-regulares ($m = \frac{3n}{4}$)
	\end{center}
\end{subfigure}
\end{figure}

\newpage
\begin{figure}[H]
\centering

\begin{subfigure}[b]{0.4\textwidth}
	\includegraphics[scale=0.6]{graph/{output_grasp_6_4_arbol.csvTime}.pdf}
	\begin{center}
	Arboles Binarios
	\end{center}
\end{subfigure}
\begin{subfigure}[b]{0.4\textwidth}
	\includegraphics[scale=0.6]{graph/{output_grasp_6_1_clique.csvTime}.pdf}
	\begin{center}
	Clique
	\end{center}
\end{subfigure}
\end{figure}

Para el analisis del tamaño de la solucion, vamos a ver los resultados por cada familia. En el caso de los aleatorios, los resultados para estas configuraciones fueron los siguiente:

\begin{table}[H]
\centering
\caption{Grafos aleatorios}
\label{my-label}
\begin{tabular}{|l|lll|}
\hline
        & \multicolumn{1}{l|}{m = n/2} & \multicolumn{1}{l|}{m = n} & m = 2n \\ \hline
n = 40  & 29                           & 21                         & 15     \\ \cline{1-1}
n = 60  & 40                           & 31                         & 21     \\ \cline{1-1}
n = 80  & 53                           & 36                         & 27     \\ \cline{1-1}
n = 100 & 62                           & 50                         & 37     \\ \cline{1-1}
n = 120 & 81                           & 56                         & 41     \\ \hline
\end{tabular}
\end{table}

\newpage
\subsubsection{GRASP7}

Los resultados temporales obtenidos fueron los siguientes:

\begin{figure}[H]
\centering

\begin{subfigure}[b]{0.4\textwidth}
	\includegraphics[scale=0.6]{graph/{output_grasp_7_1_n.csvTime}.pdf}
	\begin{center}
	Grafos Aleatorios ($m = n$)
	\end{center}
\end{subfigure}
\begin{subfigure}[b]{0.4\textwidth}
	\includegraphics[scale=0.6]{graph/{output_grasp_7_1_2n.csvTime}.pdf}
	\begin{center}
	Grafos Aleatorios ($m = 2n$)
	\end{center}
\end{subfigure}

\begin{subfigure}[b]{0.4\textwidth}
	\includegraphics[scale=0.6]{graph/{output_grasp_7_1_n2.csvTime}.pdf}
	\begin{center}
	Grafos Aleatorios ($m = \frac{n}{2}$)
	\end{center}
\end{subfigure}
\begin{subfigure}[b]{0.4\textwidth}
	\includegraphics[scale=0.6]{graph/{output_grasp_7_3_n4.csvTime}.pdf}
	\begin{center}
	Grafos Bipartitos ($\frac{n}{4}$ nodos en la segunda componente)
	\end{center}
\end{subfigure}
\end{figure}

\begin{figure}[H]
\centering

\begin{subfigure}[b]{0.4\textwidth}
	\includegraphics[scale=0.6]{graph/{output_grasp_7_3_3n4.csvTime}.pdf}
	\begin{center}
	Grafos Bipartitos ($\frac{3n}{4}$ nodos en la segunda componente)
	\end{center}
\end{subfigure}
\begin{subfigure}[b]{0.4\textwidth}
	\includegraphics[scale=0.6]{graph/{output_grasp_7_2_n4.csvTime}.pdf}
	\begin{center}
	Grafos $d$-regulares ($d = \frac{n}{4}$)
	\end{center}
\end{subfigure}

\begin{subfigure}[b]{0.4\textwidth}
	\includegraphics[scale=0.6]{graph/{output_grasp_7_2_n2.csvTime}.pdf}
	\begin{center}
	Grafos $d$-regulares ($m = \frac{n}{2}$)
	\end{center}
\end{subfigure}
\begin{subfigure}[b]{0.4\textwidth}
	\includegraphics[scale=0.6]{graph/{output_grasp_7_2_3n4.csvTime}.pdf}
	\begin{center}
	Grafos $d$-regulares ($m = \frac{3n}{4}$)
	\end{center}
\end{subfigure}
\end{figure}

\newpage
\begin{figure}[H]
\centering

\begin{subfigure}[b]{0.4\textwidth}
	\includegraphics[scale=0.6]{graph/{output_grasp_7_4_arbol.csvTime}.pdf}
	\begin{center}
	Arboles Binarios
	\end{center}
\end{subfigure}
\begin{subfigure}[b]{0.4\textwidth}
	\includegraphics[scale=0.6]{graph/{output_grasp_7_1_clique.csvTime}.pdf}
	\begin{center}
	Clique
	\end{center}
\end{subfigure}
\end{figure}

Para el analisis del tamaño de la solucion, vamos a ver los resultados por cada familia. En el caso de los aleatorios, los resultados para estas configuraciones fueron los siguiente:

\begin{table}[H]
\centering
\caption{Grafos Aleatorios}
\label{my-label}
\begin{tabular}{|l|lll|}
\hline
        & \multicolumn{1}{l|}{m = n/2} & \multicolumn{1}{l|}{m = n} & m = 2n \\ \hline
n = 40  & 26                           & 22                         & 14     \\ \cline{1-1}
n = 60  & 40                           & 31                         & 18     \\ \cline{1-1}
n = 80  & 49                           & 35                         & 26     \\ \cline{1-1}
n = 100 & 60                           & 50                         & 34     \\ \cline{1-1}
n = 120 & 75                           & 56                         & 38     \\ \hline
\end{tabular}
\end{table}

\newpage
\subsubsection{GRASP8}

Los resultados temporales obtenidos fueron los siguientes:

\begin{figure}[H]
\centering

\begin{subfigure}[b]{0.4\textwidth}
	\includegraphics[scale=0.6]{graph/{output_grasp_8_1_n.csvTime}.pdf}
	\begin{center}
	Grafos Aleatorios ($m = n$)
	\end{center}
\end{subfigure}
\begin{subfigure}[b]{0.4\textwidth}
	\includegraphics[scale=0.6]{graph/{output_grasp_8_1_2n.csvTime}.pdf}
	\begin{center}
	Grafos Aleatorios ($m = 2n$)
	\end{center}
\end{subfigure}

\begin{subfigure}[b]{0.4\textwidth}
	\includegraphics[scale=0.6]{graph/{output_grasp_8_1_n2.csvTime}.pdf}
	\begin{center}
	Grafos Aleatorios ($m = \frac{n}{2}$)
	\end{center}
\end{subfigure}
\begin{subfigure}[b]{0.4\textwidth}
	\includegraphics[scale=0.6]{graph/{output_grasp_8_3_n4.csvTime}.pdf}
	\begin{center}
	Grafos Bipartitos ($\frac{n}{4}$ nodos en la segunda componente)
	\end{center}
\end{subfigure}
\end{figure}

\begin{figure}[H]
\centering

\begin{subfigure}[b]{0.4\textwidth}
	\includegraphics[scale=0.6]{graph/{output_grasp_8_3_3n4.csvTime}.pdf}
	\begin{center}
	Grafos Bipartitos ($\frac{3n}{4}$ nodos en la segunda componente)
	\end{center}
\end{subfigure}
\begin{subfigure}[b]{0.4\textwidth}
	\includegraphics[scale=0.6]{graph/{output_grasp_8_2_n4.csvTime}.pdf}
	\begin{center}
	Grafos $d$-regulares ($d = \frac{n}{4}$)
	\end{center}
\end{subfigure}

\begin{subfigure}[b]{0.4\textwidth}
	\includegraphics[scale=0.6]{graph/{output_grasp_8_2_n2.csvTime}.pdf}
	\begin{center}
	Grafos $d$-regulares ($m = \frac{n}{2}$)
	\end{center}
\end{subfigure}
\begin{subfigure}[b]{0.4\textwidth}
	\includegraphics[scale=0.6]{graph/{output_grasp_8_2_3n4.csvTime}.pdf}
	\begin{center}
	Grafos $d$-regulares ($m = \frac{3n}{4}$)
	\end{center}
\end{subfigure}
\end{figure}

\newpage
\begin{figure}[H]
\centering

\begin{subfigure}[b]{0.4\textwidth}
	\includegraphics[scale=0.6]{graph/{output_grasp_8_4_arbol.csvTime}.pdf}
	\begin{center}
	Arboles Binarios
	\end{center}
\end{subfigure}
\begin{subfigure}[b]{0.4\textwidth}
	\includegraphics[scale=0.6]{graph/{output_grasp_8_1_clique.csvTime}.pdf}
	\begin{center}
	Clique
	\end{center}
\end{subfigure}
\end{figure}

Para el analisis del tamaño de la solucion, vamos a ver los resultados por cada familia. En el caso de los aleatorios, los resultados para estas configuraciones fueron los siguiente:

\begin{table}[H]
\centering
\caption{Grafos Aleatorios}
\label{my-label}
\begin{tabular}{|l|lll|}
\hline
        & \multicolumn{1}{l|}{m = n/2} & \multicolumn{1}{l|}{m = n} & m = 2n \\ \hline
n = 40  & 25                           & 21                         & 16     \\ \cline{1-1}
n = 60  & 40                           & 31                         & 20     \\ \cline{1-1}
n = 80  & 53                           & 36                         & 26     \\ \cline{1-1}
n = 100 & 60                           & 48                         & 34     \\ \cline{1-1}
n = 120 & 76                           & 54                         & 38     \\ \hline
\end{tabular}
\end{table}


\subsubsection{Conclusion}

Primero vamos a ver los resultados por cada familia.

\begin{itemize}
	\item Grafos Aleatorios: Para esta familia los resultados fueron variados, y muchos de ellos pudieron mejorar por un margen amplio a las otras heuristicas vistas con anterioridad, sin tomar un tiempo adicional demasiado grande. Los mejores resultados observados en termino de calidad de soluciones es el de GRASP3, que no solo dio mejor solucion en casi todos los casos, sino que ademas fue de las mas veloces.	
	\item Grafos Bipartitos: Los tiempos de ejecucion para todas las instancias de GRASP fueron en general bastante elevados para estos casos. Lamentablemente la calidad de las soluciones variaron bastante respecto a las otras heuristicas, la tendencia entre las diferentes implementaciones de todas formas fue muy marcada.
	\item Grafos $d$-regulares: Esta familia no tuvo buen rendimiento con las diferentes versiones de GRASP. Tambien los resultados obtenidos fueron peores que con las otras Heuristicas implementadas anteriormente.
	\item Arboles binarios: Al igual que con tas las otras heuristicas, el tiempo que tomo resolver cada uno de los grafos no fue constante. Respecto al tamaño de las soluciones, los resultados obtenidos tendian a alejarse de los valores ideales.
	\item Cliques: La resolucion de la cliques tomo una sorprendente cantidad de tiempo a medida que avanzaba el tiempo, esto se dio en una gran cantidad de las implementaciones, principalmente en GRASP3, GRASP4, GRASP7 y GRASP8.
\end{itemize}

Las heuristicas GRASP demostraron que habia un gran margen de mejora para los grafos aleatorios, los resultados obtenidos fueron en su mayoria mejores que los conseguidos aplicando las Heuristicas anteriores, un punto importante a destacar es que el tiempo que tomo la resolucion no fue mucho mayor al de las otras Heuristicas. Lamentablemente, la eficiencia de las diferentes configuraciones de GRASP no se mostraron en las otras familias, inlcuso llegando a casos donde los resultados fueron peores.

A pesar de todo esto, consideramos que la mejor configuracion fue GRASP3, ya que si bien esta no tuvo un buen rendimiento con las familias que no sean la aleatoria, ninguna configuracion fue particularmente buena para el resto de las familias. Valoramos el caso aleatorio principalmente, ya que consideramos que es el que mas chances tenemos de encontrar en un caso real.

\subsection{Calibracion de parametros}

Una vez elegida la configuracion, se procedio con la calibracion de parametros. Primero vamos a ver los resultados obtenidos de variar el valor de $j$, mantiendo el de $k$ en 5. Para probar cada uno de los paramteros se probo con Grafos Aleatorios (con $m = 2n$), para analizar si era posible mejorar el tiempo de convergencia. Los resultados fueron:

\begin{figure}[H]
\centering

\begin{subfigure}[b]{0.4\textwidth}
	\includegraphics[scale=0.6]{graph/{output_grasp_3_k1_1_2n.csvTime}.pdf}
	\begin{center}
	$j = 1$
	\end{center}
\end{subfigure}
\begin{subfigure}[b]{0.4\textwidth}
	\includegraphics[scale=0.6]{graph/{output_grasp_3_k2_1_2n.csvTime}.pdf}
	\begin{center}
	$j = 2$
	\end{center}
\end{subfigure}

\begin{subfigure}[b]{0.4\textwidth}
	\includegraphics[scale=0.6]{graph/{output_grasp_3_k3_1_2n.csvTime}.pdf}
	\begin{center}
	$j = 3$
	\end{center}
\end{subfigure}
\begin{subfigure}[b]{0.4\textwidth}
	\includegraphics[scale=0.6]{graph/{output_grasp_3_k4_1_2n.csvTime}.pdf}
	\begin{center}
	$j = 4$
	\end{center}
\end{subfigure}
\end{figure}

\begin{figure}[H]
\centering

\begin{subfigure}[b]{0.4\textwidth}
	\includegraphics[scale=0.6]{graph/{output_grasp_3_k5_1_2n.csvTime}.pdf}
	\begin{center}
	$j = 5$
	\end{center}
\end{subfigure}
\begin{subfigure}[b]{0.4\textwidth}
	\includegraphics[scale=0.6]{graph/{output_grasp_3_k6_1_2n.csvTime}.pdf}
	\begin{center}
	$j = 6$
	\end{center}
\end{subfigure}

\begin{subfigure}[b]{0.4\textwidth}
	\includegraphics[scale=0.6]{graph/{output_grasp_3_k7_1_2n.csvTime}.pdf}
	\begin{center}
	$j = 7$
	\end{center}
\end{subfigure}
\begin{subfigure}[b]{0.4\textwidth}
	\includegraphics[scale=0.6]{graph/{output_grasp_3_k8_1_2n.csvTime}.pdf}
	\begin{center}
	$j = 8$
	\end{center}
\end{subfigure}
\end{figure}

\begin{figure}[H]
\centering

\begin{subfigure}[b]{0.4\textwidth}
	\includegraphics[scale=0.6]{graph/{output_grasp_3_k9_1_2n.csvTime}.pdf}
	\begin{center}
	$j = 9$
	\end{center}
\end{subfigure}
\begin{subfigure}[b]{0.4\textwidth}
	\includegraphics[scale=0.6]{graph/{output_grasp_3_k10_1_2n.csvTime}.pdf}
	\begin{center}
	$j = 10$
	\end{center}
\end{subfigure}
\end{figure}

Podemos ver claramente qe si $j = 3$, obtenemos los mejores resultados. Tambien se analizo los posibles valores de $k$, manteniendo $j = 5$. Se obtuvieron los siguientes resultados:

\begin{figure}[H]
\centering

\begin{subfigure}[b]{0.4\textwidth}
	\includegraphics[scale=0.6]{graph/{output_grasp_3_1_1_2n.csvTime}.pdf}
	\begin{center}
	$k = 1$
	\end{center}
\end{subfigure}
\begin{subfigure}[b]{0.4\textwidth}
	\includegraphics[scale=0.6]{graph/{output_grasp_3_2_1_2n.csvTime}.pdf}
	\begin{center}
	$k = 2$
	\end{center}
\end{subfigure}
\end{figure}

\begin{figure}[H]
\centering

\begin{subfigure}[b]{0.4\textwidth}
	\includegraphics[scale=0.6]{graph/{output_grasp_3_3_1_2n.csvTime}.pdf}
	\begin{center}
	$k = 3$
	\end{center}
\end{subfigure}
\begin{subfigure}[b]{0.4\textwidth}
	\includegraphics[scale=0.6]{graph/{output_grasp_3_4_1_2n.csvTime}.pdf}
	\begin{center}
	$k = 4$
	\end{center}
\end{subfigure}

\begin{subfigure}[b]{0.4\textwidth}
	\includegraphics[scale=0.6]{graph/{output_grasp_3_5_1_2n.csvTime}.pdf}
	\begin{center}
	$k = 5$
	\end{center}
\end{subfigure}
\begin{subfigure}[b]{0.4\textwidth}
	\includegraphics[scale=0.6]{graph/{output_grasp_3_6_1_2n.csvTime}.pdf}
	\begin{center}
	$k = 6$
	\end{center}
\end{subfigure}

\begin{subfigure}[b]{0.4\textwidth}
	\includegraphics[scale=0.6]{graph/{output_grasp_3_7_1_2n.csvTime}.pdf}
	\begin{center}
	$k = 7$
	\end{center}
\end{subfigure}
\begin{subfigure}[b]{0.4\textwidth}
	\includegraphics[scale=0.6]{graph/{output_grasp_3_8_1_2n.csvTime}.pdf}
	\begin{center}
	$k = 8$
	\end{center}
\end{subfigure}
\end{figure}

\begin{figure}[H]
\centering

\begin{subfigure}[b]{0.4\textwidth}
	\includegraphics[scale=0.6]{graph/{output_grasp_3_9_1_2n.csvTime}.pdf}
	\begin{center}
	$k = 9$
	\end{center}
\end{subfigure}
\begin{subfigure}[b]{0.4\textwidth}
	\includegraphics[scale=0.6]{graph/{output_grasp_3_10_1_2n.csvTime}.pdf}
	\begin{center}
	$k = 10$
	\end{center}
\end{subfigure}
\end{figure}

A diferencia del caso de $j$, aqui no hubo un impacto tan grande, si bien hubo diferencias, al variar el valor de $j$, no solo cambio el tiempo de convergencia, sino que ademas se estabilizo el tiempo promedio. Consideramos que si tomamos $j = 3$ y $k = 5$ pudimos conseguir una buena configuracion, un analisis mas exhaustivo habria sido analizar las 100 posibles combinaciones tomando valores entre 1 y 10 para $j$ y $k$. Sin embargo, los resultados obtenidos fueron buenos.

\end{document}