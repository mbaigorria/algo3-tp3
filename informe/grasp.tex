\section{Metaheurística GRASP}

\subsection{Algoritmo}

GRASP (Greedy Randomized Adaptative Search Procedure) es una combinación entre una heuristica golosa aleatorizada y un procedimiento de búsqueda local. La metaheurística se puede ver con el siguiente pseudocódigo:

\begin{algorithmic}
\Procedure{grasp}{G, k}
\State{G bestSolution}
\While{$!terminationCondition(G, k, bestSolution)$}
	\State{s $\gets$ randomGreedyHeuristic(G)}
	\State{s $\gets$ localSearch(G,s)}
	\If{$|s| < |bestSolution|$}
	\State{bestSolution $\gets$ s}
	\EndIf
\EndWhile
\EndProcedure
\end{algorithmic}

De este procedimiento surgen dos preguntas, que en realidad son cosas que debemos definir. De donde proviene la aleatoriedad de la heurística greedy? Cual es criterio de terminación que utilizaremos?

\subsubsection{Random Greedy Heuristic}

\begin{enumerate}
\item Por cantidad:
Para agregarle una componente aleatoria a \texttt{GRASP}, se propone fabricar en cada paso de la heurística constructiva golosa una \textit{Lista Restricta de Candidatos} (RCL) y elegir aleatoriamente un candidato de esta lista. Para ello, decidimos crear la función \texttt{greedyHeapConstructiveRandomized(Node graph[], int n, int k)} que lo que hace es ir eligiendo los $k$ vértices con mayor grado utilizando un heap como estructura auxiliar.
\item Por valor: \textbf{TODO (programarlo también, es un toque)}
\end{enumerate}

\subsubsection{Criterio de terminación}
\begin{enumerate}
\item No se encontró ninguna mejora en las ultimas $j$ iteraciones.
\item Se alcanzo un limite prefijado de  $j$ iteraciones.
\end{enumerate}

\textbf{2x2x2, tenes 4 experimentaciones posibles combinando esto, x2 con las dos localidades definidas (hay que testear, mejorar y corregir eso)}