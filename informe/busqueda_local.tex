\section{Heurística de Búsqueda Local}

\subsection{Algoritmo}

El algoritmo propuesto sigue los siguientes pasos:

\begin{enumerate}
\item Utilizando la heuristica constructiva golosa, armamos una potencial solución factible $s \in S$. También es posible tomar la primera solución encontrada por backtracking.
\item A continuacion, construimos un posible vecino $s' \in N(s)$. Para ello, primero tomamos un nodo del cubrimiento, y lo sacamos.
\item Tomamos todos los nodos adyacentes a ese nodo, y los agregamos al conjunto solución.
\item Para mantener la independencia, quitamos todos los nodos adyacentes a los nuevos nodos.
\item A medida que recorremos los nodos, vamos marcando si los hemos manipulado durante el procedimiento.
\item Si la nueva solución $s'$ es mejor, intercambiamos la solucion anterior $s$ por $s'$ y repetimos el paso 2. Caso contrario, finaliza el algoritmo.
\end{enumerate}

\subsection{Complejidad}